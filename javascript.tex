\documentclass[a4paper,12pt]{article}


\usepackage[top = .3in,
            right = .2in,
            bottom = .3in,
            left = .2in]{geometry}
\usepackage[UTF8]{ctex}
\usepackage{verbatimbox}
\usepackage{hyperref,xcolor}
\usepackage{amsmath,amssymb}
\usepackage{paralist}
\usepackage{fancyvrb}
\usepackage{minted}
\usepackage{lipsum}



\definecolor{myblue}{rgb}{0.5, 0.0, 1.0}
\hypersetup{colorlinks=true,urlcolor=myblue,}

\newminted{js}{gobble=0,linenos=false,frame=single,breaklines=true,obeytabs=true,tabsize=2,escapeinside=\#\#}
\newminted{html}{gobble=0,linenos=false,frame=single,breaklines=true,escapeinside=@@}

\begin{document}
%\begin{CJK}{UTF8}{gbsn}
\fontsize{14pt}{15.6pt}
\selectfont

%\lipsum[1-5]

%\linespread{1}
\begin{enumerate}

\item VPS:
\begin{itemize}
  \item First,
\end{itemize}

\item \verb|setTimeout()| method
\begin{htmlcode}
<!DOCTYPE html>
<html>
<body>

<p>Click the button to wait 3 seconds, then alert "Hello".</p>

<button onclick="myFunction()">Try it</button>

<script>
var myVar;

function myFunction() {
  // alertFunc() won't work properly. In jQuery the second option is mandatory, like
  // setTimeout('$("#uploadStatusBar").hide();',1000);
  myVar = setTimeout(alertFunc, 3000);
}
// Or,
function myFunction() {
  myVar = setTimeout("alertFunc();", 3000); // semicolon can be omitted, but double quotes can not, see the comments of the first option
}

function alertFunc() {
  alert("Hello!");
}
</script>

</body>
</html>
\end{htmlcode}

\item Advanced CSS Grid
\begin{htmlcode}
<div class="container">
  <header>Welcome!</header>
  <nav>Links!</nav>
  <section class="info">Info!</section>
  <section class="services">Services!</section>
  <footer>Contact us!</footer>
</div>
\end{htmlcode}

\begin{htmlcode}
.container {
  display: grid;
  max-width: 900px;
  position: relative;
  margin: auto;
  grid-template-areas: "head head"
                       "nav nav" 
                       "info services"
                       "footer footer";
  grid-template-rows: 300px 120px 800px 120px;
  grid-template-columns: 1fr 3fr; 
}

header {
  grid-area: head;
} 

nav {
  grid-area: nav;
} 

.info {
  grid-area: info;
} 

.services {
  grid-area: services;
}

footer {
  grid-area: footer;
}
\end{htmlcode}

\item vue computed property setters (We did this by making the value of languageLevel an object with two keys, get and set, each with a value of a function). In order to add a setter to a computed property , you must make the value of the computed property an object with get and set properties (like \verb|<select v-model="languageLevel">| case)
\begin{jscode}
const app = new Vue({
  el: '#app',
  data: {
    hoursStudied: 274
  },
  computed: {
    languageLevel: {
      get: function() {
        if (this.hoursStudied < 100) {
          return 'Beginner';
        } else if (this.hoursStudied < 1000) {
          return 'Intermediate';
        } else {
          return 'Expert';
        }
      },
      set: function(newLanguageLevel) {
        if (newLanguageLevel === 'Beginner') {
          this.hoursStudied = 0;
        } else if (newLanguageLevel === 'Intermediate') {
          this.hoursStudied = 100;
        } else if (newLanguageLevel === 'Expert') {
          this.hoursStudied = 1000;
        }
      }
    }
  }
});

<div id=“app”>
  <p>You have studied for {{ hoursStudied }} hours. You have {{ languageLevel }}-level mastery.</p>
  <span>Change Level:</span>
  <select v-model="languageLevel">
    <option>Beginner</option>
    <option>Intermediate</option>
    <option>Expert</option>
  </select>
</div>
\end{jscode}

\item vue computed properties
\begin{jscode}
const app = new Vue({
  el: '#app',
  data: {
    hoursStudied: 274
  },
  computed: {
    languageLevel: function() {
      if (this.hoursStudied < 100) {
        return 'Beginner';
      } else if (this.hoursStudied < 1000) {
        return 'Intermediate';
      } else {
        return 'Expert';
      }
    }
  }
});

<div id="app">
  <p>You have studied for {{ hoursStudied }} hours. You have {{ languageLevel }}-level mastery.</p>
</div>
\end{jscode}

\item Chaining multiple promises
\begin{jscode}
app.js
const {checkInventory, processPayment, shipOrder} = require('./library.js');

const order = {
  items: [['sunglasses', 1], ['bags', 10]],
  giftcardBalance: 79.82
};

checkInventory(order)
.then((resolvedValueArray) => {
  // Write the correct return statement here:
  return processPayment(resolvedValueArray);
})
.then((resolvedValueArray) => {
  // Write the correct return statement here:
  return shipOrder(resolvedValueArray)
})
.then((successMessage) => {
  console.log(successMessage);
})
.catch((errorMessage) => {
  console.log(errorMessage);
});
\end{jscode}

\begin{jscode}
library.js
const store = {
  sunglasses: {
    inventory: 817, 
    cost: 9.99
  },
  pants: {
    inventory: 236, 
    cost: 7.99
  },
  bags: {
    inventory: 17, 
    cost: 12.99
  }
};

const checkInventory = (order) => {
  return new Promise ((resolve, reject) => {
   setTimeout(()=> {  
   const itemsArr = order.items;  
   let inStock = itemsArr.every(item => store[item[0]].inventory >= item[1]);
   
   if (inStock){
     let total = 0;   
     itemsArr.forEach(item => {
       total += item[1] * store[item[0]].cost
     });
     console.log(`All of the items are in stock. The total cost of the order is ${total}.`);
     resolve([order, total]);
   } else {
     reject(`The order could not be completed because some items are sold out.`);
   }     
}, generateRandomDelay());
 });
};

const processPayment = (responseArray) => {
  const order = responseArray[0];
  const total = responseArray[1];
  return new Promise ((resolve, reject) => {
   setTimeout(()=> {  
   let hasEnoughMoney = order.giftcardBalance >= total;
   // For simplicity we've omited a lot of functionality
   // If we were making more realistic code, we would want to update the giftcardBalance and the inventory
   if (hasEnoughMoney) {
     console.log(`Payment processed with giftcard. Generating shipping label.`);
     let trackingNum = generateTrackingNumber();
     resolve([order, trackingNum]);
   } else {
     reject(`Cannot process order: giftcard balance was insufficient.`);
   }
   
}, generateRandomDelay());
 });
};

const shipOrder = (responseArray) => {
  const order = responseArray[0];
  const trackingNum = responseArray[1];
  return new Promise ((resolve, reject) => {
   setTimeout(()=> {
     resolve(`The order has been shipped. The tracking number is: ${trackingNum}.`);
}, generateRandomDelay());
 });
};

// This function generates a random number to serve as a "tracking number" on the shipping label. In real life this wouldn't be a random number
function generateTrackingNumber() {
  return Math.floor(Math.random() * 1000000);
}

// This function generates a random number to serve as delay in a setTimeout() since real asynchrnous operations take variable amounts of time
function generateRandomDelay() {
  return Math.floor(Math.random() * 2000);
}

module.exports = {checkInventory, processPayment, shipOrder};
\end{jscode}

\item 
\begin{jscode}
init(pageNum = 1) {
	window.scrollTo(0, 0)
	const { sortPrice } = this.state
	this.setState({
		loading: true
	}, () => {
		queryTicketList({
			pageNum: pageNum,
			reorder: sortPrice || '',
			searchInput: '上海',
			cityName: '上海'
		}).then(data => {
			const { pageCount, viewInfoList } = data
			this.setState({
				loading: false,
				dataSource: viewInfoList,
				total: pageCount,
				current: pageNum
			})
		})
	})
}

paiXu = (sortPrice) => {
	this.setState({ sortPrice }, () => this.init())  //this.init()需要在this.setState()的回调函数里面执行
}
\end{jscode}

\item often used DOM properties and methods (scrollHeight, offsetHeight, clientHeight, scrollTop, offsetTop)
\begin{jscode}
// 获取纵向滚动条的滚动距离
const y = document.body.scrollTop || document.documentElement.scrollTop

document.body.innerHTML = 'This is the text of the body element';

document.body.firstChild.parentNode.innerHTML = 'I am the parent and my inner HTML has been replaced!';

document.querySelector('h1').innerHTML = 'Most popular TV show searches in 2016'; // returns the first selected element

document.querySelector('#fourth').innerHTML = 'Fourth element';

document.body.style.backgroundColor = '#201F2E';

let liRef = document.createElement('li');
liRef.id = 'oaxaca';
liRef.innerHTML = "Oaxaca, Mexico";

document.getElementById('more-destinations').appendChild(liRef);

parent.removeChild(child);
\end{jscode}

\item webpack.config.js
\begin{jscode}
const path = require('path');
const HtmlWebpackPlugin = require('html-webpack-plugin');

// common.js syntax
module.exports = {
  entry: './src/index.js',
  output: {
    path: path.join(__dirname, '/dist'),
    filename: 'index_bundle.js'
  },
  module: {
    rules: [
      {
        test: /\.js$/,
        exclude: /node_modules/,
        use: {
          loader: 'babel-loader'
        }, Or
        use: [
          // apply multiple loaders and options
          "htmllint-loader",
          {
            loader: "html-loader",
            options: {
              /* ... */
            }
          }
        ]
      }
    ]
  },
  plugins: [
    new HtmlWebpackPlugin({
      template: './src/index.html',
      filename: 'index.html',
      inject: 'body'
    })
  ]
}
\end{jscode}

\item Computed property names

Starting with ECMAScript 2015, the object initializer syntax also supports computed property names. That allows you to put an expression in brackets \verb|[]|, that will be computed and used as the property name.
\begin{jscode}
// Computed property names (ES2015)
var i = 0;
var a = {
  ['foo' + ++i]: i,
  ['foo' + ++i]: i,
  ['foo' + ++i]: i
};

console.log(a.foo1); // 1
console.log(a.foo2); // 2
console.log(a.foo3); // 3

var param = 'size';
var config = {
  [param]: 12,
  ['mobile' + param.charAt(0).toUpperCase() + param.slice(1)]: 4
};

console.log(config); // {size: 12, mobileSize: 4}
\end{jscode}

\item A presentational component can often be written as a \textit{stateless functional component}

\begin{jscode}
// A component class written in the usual way:
export class MyComponentClass extends React.Component {
  render() {
    return <h1>Hello world</h1>;
  }
}

// The same component class, written as a stateless functional component:
export const MyComponentClass = () => {
  return <h1>Hello world</h1>;
}

// Works the same either way:
ReactDOM.render(
	<MyComponentClass />,
	document.getElementById('app')
);
\end{jscode}

\item Child Components Update Their Parents' state in React

Parent.js
\begin{jscode}
import React from 'react';
import ReactDOM from 'react-dom';
import { Child } from './Child';

class Parent extends React.Component {
  constructor(props) {
    super(props);
    this.changeName = this.changeName.bind(this);
    this.state = { name: 'Frarthur' };
  }
  
  changeName(newName) {
    this.setState({ name: newName });
  }

  render() {
    return <Child name={this.state.name} onChange={this.changeName} />
  }
}

ReactDOM.render(
	<Parent />,
	document.getElementById('app')
);
\end{jscode}
You cannot declare method \verb|changeName(newName)| as \verb|changeName: function(newName)|, otherwise won't work. The same goes for \verb|render()|\\


Child.js
\begin{jscode}
import React from 'react';

export class Child extends React.Component {
  constructor(props) {
    super(props);
    this.handleChange = this.handleChange.bind(this);
  }
  
  handleChange(e) {
    const name = e.target.value;
    this.props.onChange(name);
  }
  
  render() {
    return (
      <div>
        <h1>
          Hey my name is {this.props.name}!
        </h1>
        <select id="great-names" onChange={this.handleChange}>
          <option value="Frarthur">
            Frarthur
          </option>

          <option value="Gromulus">
            Gromulus
          </option>

          <option value="Thinkpiece">
            Thinkpiece
          </option>
        </select>
      </div>
    );
  }
}
\end{jscode}


\item class inheritance
\begin{jscode}
class HospitalEmployee {
  constructor(name) {
    this._name = name;
    this._remainingVacationDays = 20;
  }
  
  get name() {
    return this._name;
  }
  
  get remainingVacationDays() {
    return this._remainingVacationDays;
  }
  
  takeVacationDays(daysOff) {
    this._remainingVacationDays -= daysOff;
  }
}

class Nurse extends HospitalEmployee {
  constructor(name, certifications) {
    super(name);
    this._certifications = certifications;
  }  
  
  get certifications() {
    return this._certifications;
  }
  
  addCertification(newCertification) {
    this.certifications.push(newCertification);
  }
}

const nurseOlynyk = new Nurse('Olynyk', ['Trauma','Pediatrics']);
nurseOlynyk.takeVacationDays(5);
console.log(nurseOlynyk.remainingVacationDays);
nurseOlynyk.addCertification('Genetics');
console.log(nurseOlynyk.certifications);
\end{jscode}

\item The major difference between a GET request and POST request is that a POST request requires additional information to be sent through the request. This additional information is sent in the body of the post request.

\item \textbf{async await POST}
\begin{jscode}
// async await POST

async function getData(){
  try {
    const response = await fetch('http://api-to-call.com/endpoint', { // sends request
      method: 'POST',
      body: JSON.stringify({id: '200'})
    });
    if (response.ok){ // handles response if successful
      const jsonResponse = await response.json();
      // Code to execute with jsonResponse
    }
    throw new Error('Request Failed!');
  } catch (error){ // handles response if unsuccessful
    console.log(error);
  }
}
\end{jscode}

\item \textbf{async await GET}
\begin{jscode}
// async await GET

async function getData(){
  try {
    const response = await fetch('http://api-to-call.com/endpoint');
    if (response.ok){ // handles response if successful
      const jsonResponse = await response.json();
      // Code to execute with jsonResponse
    }
    throw new Error('Request Failed!');
  } catch (error) { // handles response if unsuccessful
    console.log(error);
  }
}
\end{jscode}

\item 
\begin{itemize}
\item used \verb|fetch()| to make GET and POST requests
\item check the status of the responses coming back
\item catch errors that might possibly arise
\item taking successful responses and rendering it on the webpage
\end{itemize}

\item \textbf{fetch() POST Requests}
\begin{jscode}
// fetch POST

fetch('http://api-to-call.com/endpoint', {
  method: 'POST',
  body: JSON.stringify({id: '200'}) // sends request
}).then(response => {
  if (response.ok){
    return response.json(); // converts response object to JSON
  }
  throw new Error('Request failed!');
}, networkError => console.log(networkError.message) // handles errors
).then(jsonResponse => {
  // Code to execute with jsonResponse // handles success
});
\end{jscode}

\item \textbf{fetch() GET Requests}
\begin{jscode}
// fetch GET

fetch('http://api-to-call.com/endpoint').then(response => { // sends request
  if (response.ok){
    return response.json(); // converts response object to JSON
  }
  throw new Error('Request failed!');
}, networkError => console.log(networkError.message) // handles errors
).then(jsonResponse => {
  // Code to execute with jsonResponse // handles success
});
\end{jscode}

\item Boilerplate code for making an XHR POST request\\
From codecademy:
\begin{jscode}
// XMLHttpRequest POST

const xhr = new XMLHttpRequest();
const url = 'http://api-to-call.com/endpoint';
const data = JSON.stringify({id: '200'});  // Converts data to a JSON string

// handles response
xhr.responseType = 'json';
xhr.onreadystatechange = () => {
  if (xhr.readyState === XMLHttpRequest.DONE){
    // Code to execute with response
  }
};

xhr.open('POST', url);
xhr.send(data);
\end{jscode}

\item Boilerplate code for making an XHR GET request\\
From codecademy:
\begin{jscode}
// XMLHttpRequest GET

const xhr = new XMLHttpRequest(); // creates new object
const url = 'http://api-to-call.com/endpoint';

// handle responses
xhr.responseType = 'json';
xhr.onreadystatechange = () => {
  if(xhr.readyState === XMLHttpRequest.DONE){
    // Code to execute with response
  }
};

//opens request and sends object
xhr.open('GET', url);
xhr.send();
\end{jscode}
From w3schools
\begin{jscode}
var xhttp = new XMLHttpRequest();
xhttp.onreadystatechange = function() {
    if (this.readyState == 4 && this.status == 200) {
       // Typical action to be performed when the document is ready:
       document.getElementById("demo").innerHTML = xhttp.responseText;
    }
};
xhttp.open("GET", "filename", true);
xhttp.send();
\end{jscode}

\item With a GET request, we're retrieving, or \textit{getting}, information from some source (usually a website). For a POST request, we're \textit{posting} information to a source that will process the information and send it back

\item json example. \textbf{Data types}: Number, String, Boolean, Array, Object, Null
\begin{jscode}
{
	"name": "Brad Traversy",
	"age": 35,
	"address": {
		"street": "5 main st",
		"city": "Boston"
	},
	"children": ["Brianna", "Nicholas"]
}
\end{jscode}

\item javascript object example
\begin{jscode}
var person = {
  name: "Brad",
  age: 35,
  email: function(){
    return 'brad@gmail.com';
  }
};
console.log(person.name);
console.log(person.email());
\end{jscode}

\item Section \textbf{DOM Template Parsing Caveats} covered \verb|is| is special attribute offers a workaround

\item Wrap form widgets inside a \verb|p| tag
\begin{htmlcode}
<p>
  <label for="name">
    <span>Name: </span>
    <strong><abbr title="required">*</abbr></strong>
  </label>
  <input type="text" id="name" name="username">
</p>
\end{htmlcode}

\item Three ways to change elements' visibility
\begin{htmlcode}
display: none; /* completely gone, never existed */
visibility: hidden; /* still occupies space */
opacity: 0; /* still occupies space */
\end{htmlcode}

\item CSS Margins: You can set the margin property to \verb|auto| to horizontally center the element within its container. To horizontally center a block element (like $\langle$div$\rangle$), use \verb|margin: auto;| \textbf{Note:} Center aligning has no effect if the width property is not set (or set to 100\%).

To horizontally center an image use\\ \verb|<img style="display: block; margin: 8px auto 15px auto;"|\\
\verb|display: block;| 一定要加

\item The default font-size is 16px, so \verb|1em| equals 16px

\item \verb|width| and \verb|height| can only be applied to elements that are not inline elements. Some examples of inline and block elements, also see \href{https://www.w3schools.com/html/html_blocks.asp}{w3schools} (块级元素 内联元素)
\begin{htmlcode}
<section id="inline">
  <span>inline</span>
  <a>inline</a>
  <b>inline</b>
  <em>inline</em>
</section>
<section id="block">
  <div>block</div>
  <nav>nav</nav>
  <aside>main</aside>
  <main>main</main>
</section>
\end{htmlcode}

\item \verb|<h1>| is the most important part of a html doc

\item self-closing tags in HTML5: \verb|<br> <embed> <hr> <iframe> <img> <input> <link>| \verb|<meta>|, closing forward slash is optional

\item Layout elements
\begin{htmlcode}
<body>
  <header>
    <nav>

    </nav>
  </header>
  <section>
    <main>
      <article>

      </article>
    </main>
    <aside>

    </aside>
  </section>
  <footer>

  </footer>
</body>
\end{htmlcode}

\item In addition to data properties, Vue instances expose a number of useful instance properties and methods. These are prefixed with \$ to differentiate them from user-defined properties

\item Out of the box, webpack won't require you to use a configuration file. However, it will assume the entry point of your project is \verb|src/index.js| and will output the result in \verb|dist/main.js| minified and optimized for production

\item When installing a package that will be bundled into your production bundle, you should use \verb|npm install --save|. If you're installing a package for development purposes (e.g. a linter, testing libraries, etc.) then you should use \verb|npm install --save-dev|

\item Popular CSS pre-processors including LESS, SASS, Stylus, and PostCSS

\item Follow this guide if the built-in configuration of Vue CLI does not suit your needs, or you'd rather create your own webpack config from scratch
	
\item \verb|vue-loader| is a loader for \textbf{webpack} that allows you to author Vue components in a format called\textbf{ Single-File Components (SFCs)}
	
\item \verb|http.request()| returns an instance of the \verb|http.ClientRequest| class. The\\ \verb|ClientRequest| instance is a writable stream. If one needs to upload a file with a POST request, then write to the \verb|ClientRequest| object.	
	
\item  With \verb|http.request()| one must always call \verb|req.end()| to signify the end of the request -- even if there is no data being written to the request body	
	

\item \verb|querystring.parse(str[, sep[, eq[, options]]])| parses a URL query string (\verb|str|) into a collection of key and value pairs. For example, the query string 'foo=bar\&abc=xyz \&abc=123' is parsed into:
\begin{jscode}
{
	foo: 'bar',
	abc: ['xyz', '123']
}	
\end{jscode}

\item \verb|querystring.stringify(obj[, sep[, eq[, options]]])| produces a URL query string from a given obj by iterating through the object's ``own properties"
\begin{jscode}
querystring.stringify({ foo: 'bar', baz: ['qux', 'quux'], corge: '' });
// returns 'foo=bar&baz=qux&baz=quux&corge='	
\end{jscode}

\item \verb|JSON.stringify(value[, replacer[, space]])| converts a JavaScript value to a JSON string
\begin{jscode}
let person = {
  name: "Brad",
  age: 35
};

person = JSON.stringify(person);
// person = JSON.parse(person);  // back to an object

console.log(person);
\end{jscode}

\item \verb|JSON.parse(text[, reviver])| parses a JSON string, constructing the JavaScript value or object described by the string. trailing commas are not allowed,\\ \verb|JSON.parse('[1, 2, 3, 4, ]');| will throw an error

\item If you access a method \textbf{without ()}, it will return the \textbf{function definition}

\item JS index position starts at zero!
\end{enumerate}

\begin{description}
\item Anatomy of an HTTP transaction

server.js:
\begin{jscode}
const http = require('http');

http.createServer((request, response) => {
	console.log(request.method);
	console.log(request.url);
	
	request.on('error', (err) => {
		console.error(err);
		response.statusCode = 400;
		response.end();
	});
	response.on('error', (err) => {
		console.error(err);
	});
	if(request.method === 'POST' && request.url === '/echo'){
		// let body = [];
		// request.on('data', (chunk) => {
		//     body.push(chunk);
		// }).on('end', () => {
		//     body = Buffer.concat(body).toString();
		//     response.writeHead(200, {'Content-Type': 'text/plain'});
		//     response.end(body);
		// });
		request.pipe(response);
	}else {
		response.statusCode = 404;
		response.end();
	}
}).listen(8080);

console.log('Server listening on port 8080');
\end{jscode}	

client.js
\begin{jscode}
var http = require('http');
var querystring = require('querystring');

var postData = querystring.stringify({
	'msg': 'hello world!'
});

var options = {
	hostname: 'localhost',
	port: 8080,
	method: 'POST',
	headers: {
		'Content-Type': 'application/x-www-form-urlencoded',
		'Content-Length': postData.length
	},
	agent: false,
	path: '/echo'
};

var req = http.request(options, function (res) {// function emitted when a response is received to this request
	//res is of type <http.IncomingMessage> and can be used to access response status, headers and data.
	//<http.IncomingMessage> implements Readable Stream interface
	console.log('STATUS: ' + res.statusCode);
	console.log('HEADERS: ' + JSON.stringify(res.headers));
	res.setEncoding('utf8');
	
	// get data as chunks (stream or buffer)
	res.on('data', function (chunk) {
		console.log('BODY: ' + chunk);
	});
	
	// end response
	res.on('end', function () {
		console.log('No more data in response.')
	});
});

req.on('error', function (e) {
	// console.log('problem with request: ' + e.message);
	console.error(e.stack);
});

// write data to request body
req.write(postData, 'utf8');

req.end(); //With http.request() one must always call req.end() to signify the end of the request	
\end{jscode}
	
\item {[2-5]} to \textbackslash cite\{2,3,4,5\}
\begin{jscode}
const input = "[2-5]";

var numRangeArr = input.match(/\d/gm);

var len = numRangeArr[1] - numRangeArr[0];

var resArr = [];

for(var i = 0; i <= len; i++){
	var res = Number(numRangeArr[0])+ i;
	resArr[i] = res;
	console.log(resArr);
}

var midRes = resArr.toString();

var result = "\\cite{" + midRes + "}";
\end{jscode}

\item Anonymous function:
\begin{jscode}
var myArray = ["Sam", "Mark", "Tim", "Sam"];

/*anonymous function*/
var result = myArray.filter(function (value, index, array){return array.indexOf(value) == index;});

document.write(result);
\end{jscode}
\begin{jscode}
/*let add = function(a,b){
	return a + b;
}

let multiply = function(a,b){
	return a * b;
}*/

let calc = function(num1, num2, callback){
	return callback(num1, num2);
}

console.log(calc(1, 2, function(a, b){
	return a-b;
}));
\end{jscode}


\item Factory pattern:
\begin{jscode}
var peopleFactory = function(name, age, state){

	var temp = {};
	//var temp = new Object();

	temp.age = age;
	temp.name = name;
	temp.state = state;

	temp.printPerson = function(){
		console.log(this.name + ", " + this.age + ", " + this.state);
	}

	return temp;
}

var person1 = peopleFactory("john", 23, "CA");
var person2 = peopleFactory("kim", 27, "SC");

person1.printPerson();
person2.printPerson();
\end{jscode}

\item Constructor pattern
\begin{jscode}
var peopleConstructor = function(name, age, state){

	this.name = name;
	this.age = age;
	this.state = state;

	this.printPerson = function(){
		console.log(this.name + ", " + this.age + ", " + this.state);
	}
}

var person1 = new peopleConstructor("john", 23, "CA");
var person2 = new peopleConstructor("kim", 27, "SC");

person1.printPerson();
person2.printPerson();
\end{jscode}

\item Prototype pattern
\begin{jscode}
var peopleProto = function(){

}

//prototype properties
peopleProto.prototype.age = 0;
peopleProto.prototype.name = "no name";
peopleProto.prototype.city = "no city";

peopleProto.prototype.printPerson = function(){
	console.log(this.name + ", " + this.age + ", " + this.city);
}

var person1 = new peopleProto();
person1.name = "John";
person1.age = 23;
person1.city = "CA";

console.log("name" in person1);
console.log(person1.hasOwnProperty("name"));

person1.printPerson();
\end{jscode}

\item Dynamic prototype pattern
\begin{jscode}
//dynamic prototype pattern
var peopleDynamicProto = function(name, age, state){
	this.age = age;
	this.name = name;
	this.state = state;

	// create function only once
	if(typeof this.printPerson !== "function"){
		peopleDynamicProto.prototype.printPerson = function(){
			console.log(this.name + ", " + this.age + ", " + this.state);
		}
	}
}

var person1 = new peopleDynamicProto("John", 24, "CA");
var person2 = new peopleDynamicProto("Yu", 23, "ZJ");

console.log("name" in person1);
console.log(person1.hasOwnProperty("name"));

person1.printPerson();
person2.printPerson();
\end{jscode}

\item Closure:
\begin{jscode}
var addTo = function(passed){

	var add = function(inner){
		return passed + inner;
	}

	return add;
}

var addTwo = addTo(2);
var addThree = addTo(3);

//console.dir(addTwo);
//console.dir(addThree);

console.log(addTwo(1));
console.log(addThree(1));
\end{jscode}

\item callback function: A callback is a function that is passed as an argument to another function and is executed after its parent function has completed
\begin{jscode}
let x = function(){
	console.log("i am called from inside a function");
}

let y = function(callback){
	console.log("do something");
	callback();
}

y(x);
\end{jscode}
\begin{jscode}
/*let calc = function(num1, num2, calcType){

	if(calcType === "add"){
		return num1 + num2;
	}else if(calcType === "multiply"){
		return num1 * num2;
	}

}

console.log(calc(1, 2, "multiply"));*/

let add = function(a,b){
	return a + b;
}

let multiply = function(a,b){
	return a * b;
}

let doWhatever = function(a,b){
	console.log("Here are the two numbers: ", a + "," + b);
}

let calc = function(num1, num2, callback){
	if(typeof callback === "function"){
		return callback(num1, num2);
	}
}

console.log(calc(1, 10, add));
\end{jscode}

\begin{jscode}
var myArr = [{
	num: 5,
	str: "apple"
},{
	num: 7,
	str: "cabbage"
},{
	num: 1,
	str: "ban"
}];

//anonymous function
myArr.sort(function(val1, val2){
	if(val1.str < val2.str){
		return -1;
	}else{
		return 1;
	}
})

console.log(myArr);
\end{jscode}

\item promises
\begin{jscode}
let promiseToCleanTheRoom = new Promise(function(resolve, reject){
	//cleaning the room
	let isClean = false;

	if(isClean){
		resolve("Cleaned up");
	}else{
		reject("not clean");
	}
})

promiseToCleanTheRoom.then(function(fromeResolve){
	console.log("The room is " + fromeResolve);
}).catch(function(fromReject){
	console.log("The room is " + fromReject);
})
\end{jscode}
\begin{jscode}
let cleanRoom =function(){
	return new Promise(function(resolve, reject){
		resolve("CLeaned the room ");
	})
}

let removeGarbage = function(message){
	return new Promise(function(resolve, reject){
		resolve(message + "Remove garbage ");
	})
}

let winIcecream = function(message){
	return new Promise(function(resolve, reject){
		resolve(message + "Won icecream");
	})
}

cleanRoom().then(function(result){
	return removeGarbage(result);
}).then(function(result){
	return winIcecream(result);
}).then(function(result){
	console.log("Finished " + result);
})

//do everything in parallel
/*Promise.all([cleanRoom(), removeGarbage(), winIcecream()]).then(function(){
	console.log("All finished");
})*/

//any one of them
/*Promise.race([cleanRoom(), removeGarbage(), winIcecream()]).then(function(){
	console.log("One of them is finished");
})*/
\end{jscode}

\item call, apply and bind
\begin{jscode}
var obj = {num:3};
var addToThis = function(a, b, c){
	return this.num + a + b + c;
}

// call
console.log(addToThis.call(obj, 1, 2, 3));

// apply
var arr = [1,2,3]; // only difference from `call'
console.log(addToThis.apply(obj, arr));

// bind
var bound = addToThis.bind(obj);
console.log(bound(1, 2, 3));
\end{jscode}

\item prototype inheritance
\begin{jscode}
var x = function(j){
	this.i = 0;
	this.j = j;

	this.getJ = function(){
		return this.j;
	}
}

x.prototype.getJ = function(){
	return this.j;
}

var x1 = new x(1);
var x2 = new x(4);

console.log(x1.getJ()); // use the method from the parent class, intead of creating one of own
console.log(x2.getJ()); // use the method from the parent class, intead of creating one of own
\end{jscode}

\begin{jscode}
// baseclass
var Job = function(){
	this.pays = true;
}

// prototype method
Job.prototype.print = function(){
	console.log(this.pays ? 'Please hire me' : 'no thank you');
}

// subclass
var TechJob = function(title, pays){
	Job.call(this); // inherits properties and methods from Job function

	this.title = title;
	this.pays = pays;
}

TechJob.prototype = Object.create(Job.prototype); // inherits from the prototype of Job
TechJob.prototype.constructor = TechJob; // set a constructor for TechJob

TechJob.prototype.print = function(){
	console.log(this.pays ? this.title + ' job is great, please hire me' : 'I would rather learn Javascript');
}

var softwarePosition = new TechJob('Javascript Programmer', true);
var softwarePosition2 = new TechJob('vb Programmer', false);

console.log(softwarePosition.print());
console.log(softwarePosition2.print());
\end{jscode}

\item HTML codes
\begin{htmlcode}
<textarea name="text" row="5000" cols="100" id="inputtext" style="width:1200px; height:300px; background-color: rgb(204,232,204); border: 2px solid Tomato; font-size: 15px"></textarea>
\end{htmlcode}

\end{description}







%\end{CJK}
\end{document}

