\documentclass[a4paper, 12pt]{article}

\usepackage[UTF8]{ctex}
\usepackage{graphicx}
\usepackage{comment}
\usepackage{lipsum}
\usepackage[colorlinks=true]{hyperref}
\usepackage[margin=.5in]{geometry}

\begin{document}
\pagenumbering{gobble}

\large
\title{Web dev}
\author{Joshua Yu}
\date{29 Aug. 2018}
\maketitle
\tableofcontents

%\lipsum[1-5]

\section{Project feature request and problems}
\begin{enumerate}

\item 怎样知道priceArr的数据结构 (\verb|E:\eastSun\ylx-pc-beta\src\components|\linebreak \verb|\LowPriceCalendar\index.js|)
\begin{verbatim}
priceArr
Array(0) [, …]
[[StableObjectId]]: 6
2019-02-02: Object {price: 57, surplusStock: 99}
2019-02-09: Object {price: 57, surplusStock: 99}
2019-02-16: Object {price: 57, surplusStock: 99}
2019-02-23: Object {price: 57, surplusStock: 99}
2019-03-02: Object {price: 57, surplusStock: 99}
2019-03-09: Object {price: 57, surplusStock: 99}
2019-03-16: Object {price: 57, surplusStock: 99}
2019-03-23: Object {price: 57, surplusStock: 99}
2019-03-30: Object {price: 57, surplusStock: 99}
length: 0
__proto__: Array(0) [, …]
\end{verbatim}

\item 与后台进行数据交互时,注意判断是字符串还是数字,比如 \verb|needEmail === "1"|。``0" 也为 true

\item 使用localStorage的时候,打开一个标签然后另外再打开一个标签,则第一个标签里面的内容会变成第二个标签的内容

\item 微信门票列表搜索框,当需要对历史搜索里面的数据进行修改时,点击输入框后文字消失

\item 虽然前端的密码输入为不可见,但是网络请求登录的接口里面可以看到密码

\item \href{https://ant.design/components/select-cn/}{Select选择器}里面的 onChange 和 onSelect 方法是一样的吗?

\end{enumerate}

\section{Things to be done before commit}
\begin{enumerate}

\item 如果发现了重要知识点,可以单独提取出来commit并在msg后面加上* (可单独stage那个点,可能需要备份相应的文件),*前面可空一格以防跟前面的文字连在一起难以辨认

\item Indent using tabs, and set tab size to 2

\item Delete \verb|console.log()|

\item Format document by pressing \verb|shift+alt+f| in VSCode

\end{enumerate}

\section{Caveats}
\begin{enumerate}

\item 了解后台传的数据结构对前端开发应该也有一定的帮助

\item 当需要对表格里的时间进行搜索的时候(比如“检测时间(开始--结束)”),只需要添加一个时间控件(RangePicker)里面写上“开始”和“结束” placeholder 即可(因为“检测时间”那一列是在 table 里的)

\item 当在手机微信端访问项目时,如果打不开页面可以试着关闭电脑防火墙试一下,或者wifi忘记重新连一下,然后在“允许其他设备连接”选择是

\item 前端form里面的数据类型为字符串,但是后台spring会自动进行对应类型的转换

\item 改完数据库后,重新启动后台或者前台登出登录一下

\item \verb|<link href="main.css" type="text/css" rel="stylesheet" />|\\
\verb|<meta charset="UTF-8">|

\item 在create-react-app里面安装antd后,引入使用antd会报错,需要删除\\ \verb|node_modules| 文件夹,然后重新\verb|yarn or npm i|

\item \verb|ctrl+shift+delete| 清chrome缓存

\item 以后需要避免测试人员回家了,我还有bug没有解决

\item 门票提交订单右半部分高度不随左半部分出游人增多而增加,可以新增一个div

\item codecademy 里面少一个标点符号或者拼写错误都会导致编译不通过

\item 找错的时候,首先应该用chrome developer tool确定位置

\item 
\begin{verbatim}
let tourists = 'John,'
tourists.split(',').map(item => ({name: item }))
\end{verbatim}
will return an array containing two objects
\begin{verbatim}
[{name: 'John'}, {name: ''}]
\end{verbatim}

\item 如果当前页面有从localStorage里面取数据,那么就不能直接在url上面写参数 (比如知道某个goodsTicketId,然后直接加到url后面),因为请求的数据会跟从localStorage里面取的数据不一样导致混淆

\item \verb|dangerouslySetInnerHTML| 里面的样式是用原生css写

\item 注意将less文件导入到js文件中

\item chrome和firefox浏览器之间可能有兼容性问题,比如用css属性设置文字超过四行用省略号

\item When debugging in VSCode, if you want a variable live updating then add it to \verb|watch|, there's also live updating in the \verb|debug console| (the `create live expression')

\item 数据在loading完成之前是undefined,所以不能对其进行诸如\\ \verb|address.constructor === Object|的操作 (不能读取其constructor property)

\item lineHeight应该比fontSize大1

\item When a breakpoint is not hitted in VSCode, you may need to stop by pressing \verb|shift+F5| and then restart by pressing \verb|F5|, and also if there's a new browser tab open then the breakpoint won't be hitted. You can try copy the path url into a new tab and see

\item 使用iconfont的时候,不知道怎么引用到项目中,可以点击“下载至本地”里面包含一个使用demo

\item You can set a breakpoint in VSCode, then add a \verb|Watch| in chrome, because chrome has auto-complete feature. The debug config file \verb|launch.json|'s url needs to be set according to the running port

\item minimize window by shortcut: \verb|alt+space+n|

\item 用chrome来调试的时候,可能需要用到\verb|_this|来获取一些变量或对象等的值

\item Postman需要开启全局代理才能请求到数据

\item chrome inspect: margin 红 \quad padding 绿 \quad content 蓝

\item Maybe you need to set the breakpoint inside the body of an arrow function in order to be hit

\item After configured webpack.config.js file, you may need to recompile to make it work

\item Chrome developer tools: open the \verb|Elements| tab to inspect the compiled html, 当判断某个元素 \verb|:hover| 时,如果当前元素搜索不到 hover 则可能在当前元素的子元素里

\item \verb|import| vs \verb|require|: import can bring in only the desired function
\begin{verbatim}
import {countItems} from ``math_array_functions''
\end{verbatim}

\item \verb|<meta name="viewport" content="width= device-width, initial-scale= 1">|

\item  FTP is the primary method by which people upload files to webservers. However, services like GoDaddy and others often try to provide simpler ways to upload files.

\item There is a space between \verb|body| and \verb|asterisk|
\begin{verbatim}
body, body *{
    margin: unset;
    padding: unset;
}
\end{verbatim}

\item Sometimes you have to close the page and then reload from webstorm to make the change take place
\end{enumerate}


\section{Git}

\subsection{Caveats}
\begin{itemize}

\item cmd, github desktop and gitee 的 commit 看的效果是一样的;cmd 的 stash 和 github desktop and gitee 的 commit 看的效果是一样的;cmd 和 github desktop 的 changes 看的效果是一样的,VSCode 看的 changes 更明白,会显示一行里面哪个部分不一样

\item \verb|warning: LF will be replaced by CRLF in| windows中的换行符为 CRLF, 而在linux下的换行符为LF,所以在执行add . 时出现提示,解决办法 (不在当前git目录执行):\\
\verb|git config --global core.autocrlf false|
\end{itemize}

\subsection{Basic Git Workflow}
\noindent\verb|git init|\\
\verb|git status| inspects the contents of the working directory and staging area\\
\verb|git add filename_1 filename_2| add files/changes to the staging area. \verb|git add .| or \verb|git add -A| shortcut for adding all the files to the staging area \\
\verb|git diff filename| press \verb|q| on your keyboard to exit diff mode. If no content has been changed then nothing will appear. Diff between \textit{add} and changes after \textit{add} \\ 
\verb|git commit -m "Complete first line of dialogue"| Standard Conventions for Commit Messages:
\begin{itemize}
\item Must be in quotation marks
\item Written in the present tense
\item Should be brief (50 characters or less) when using \verb|-m|
\end{itemize}
\verb|git log| Commits are stored chronologically in the repository and can be viewed using this command

\subsection{How to Backtrack in Git}
\subsubsection{head commit}
In Git, the commit you are currently on is known as the \verb|HEAD| commit. In many cases, the most recently made commit is the \verb|HEAD| commit

To see the HEAD commit, enter: \verb|git show HEAD|

The output of this command will display everything the \verb|git log command| displays for the \verb|HEAD| commit, plus all the file changes that were committed

\subsubsection{git checkout}
What if you decide to change the ghost's line in the working directory, but then decide you wanted to discard that change?\\
\verb|git checkout HEAD filename| or \verb|git checkout -- filename| will restore the file in your working directory to look exactly as it did when you last made a commit

Here, \verb|filename| again is the actual name of the file. If the file is named \verb|changes.txt|, the command would be\\
\verb|git checkout HEAD changes.txt|


\subsubsection{git reset I}

What if, before you commit, you accidentally delete an important line from \verb|scene-2.txt|? Unthinkingly, you add \verb|scene-2.txt| to the staging area. The file change is unrelated to the Larry/Laertes swap and you don't want to include it in the commit

We can unstage that file from the staging area using \verb|git reset HEAD filename|

\subsubsection{git reset II}
Git enables you to rewind to the part before you made the wrong turn. You can do this with:\\ 
\verb|git reset commit_SHA|

This command works by using the first 7 characters of the SHA of a previous commit. For example, if the SHA of the previous commit is\\ \verb|5d692065cf51a2f50ea8e7b19b5a7ae512f633ba|, use:\\
\verb|git reset 5d69206|\\
\verb|HEAD| is now set to that previous commit

\textbf{Before reset:} \verb|HEAD| is at the most recent commit

\textbf{After resetting:} \verb|HEAD| goes to a previously made commit of your choice. You have in essence rewound the project's history

\textbf{Then} you may want to discard all the changes in that commit with\\ \verb|git checkout HEAD filename|



\subsection{Git Branching}
Up to this point, you've worked in a single Git branch called \verb|master|. Git allows us to create branches to experiment with versions of a project. Imagine you want to create version of a story with a happy ending. You can create a new branch and make the happy ending changes to that branch only. It will have no effect on the \verb|master| branch until you're ready to merge the happy ending to the master branch

\begin{itemize}
\item You can use following command to answer the question: “which branch am I on?”\verb|git branch|

\item To create a new branch, use: \verb|git branch new_branch|. Also, branch names can't contain whitespaces: \verb|new-branch| and \verb|new_branch| are valid branch names, but \verb|new branch| is not

\item You can switch to the new branch with: \verb|git checkout branch_name| (notice the * is now over the new branch)

\item Once you switch branch, you now able to make commits on the branch that have no impact on \verb|master|

\end{itemize}

\subsubsection{git merge}
\begin{itemize}
\item Switch to master branch with \verb|git checkout master|

\item Merge new branch to master branch using \verb|git merge new_branch|

\item The merge is a "fast forward" because Git recognizes that \verb|fencing| (created new branch) contains the most recent commit. Git fast forwards \verb|master| to be up to date with \verb|fencing|

\end{itemize}

\subsubsection{delete branch}
In Git, branches are usually a means to an end. You create them to work on a new project feature, but the end goal is to merge that feature into the master branch. After the branch has been integrated into master, it has served its purpose and can be deleted

The command \verb|git branch -d branch_name| will delete the specified branch from your Git project. If some feature branches were never merged into \verb|master|, then use the uppercase D, like \verb|git branch -D branch_name|

\subsubsection{Conclusion}
\begin{itemize}
\item \verb|git branch|: Lists all a Git project's branches

\item \verb|git branch branch_name|: Creates a new branch

\item \verb|git checkout branch_name|: Used to switch from one branch to another

\item \verb|git merge branch_name|: Used to join file changes from one branch to another

\item \verb|git branch -d branch_name|: Deletes the branch specified

\end{itemize}

\subsection{Git Teamwork}
\subsubsection{git clone}
\verb|git clone remote_location clone_name|

\begin{itemize}
\item \verb|remote_location| tells Git where to go to find the remote. This could be a web address, or a filepath

\item \verb|clone_name| is the name you give to the directory in which Git will clone the repository

\end{itemize}

For example, you're collaborating with Sally, the Git remote Sally started is called: \verb|science-quizzes|

\verb|git clone science-quizzes my-quizzes| 

my-quizzes is your local copy of the science-quizzes Git project. If you commit changes to the project here, Sally will not know about them

\subsubsection{git remote -v}
Nice work! We have a clone of Sally's remote on our computer. One thing that Git does behind the scenes when you clone \verb|science-quizzes| is give the remote address the name origin, so that you can refer to it more conveniently. In this case, Sally's remote is origin.

You can see a list of a Git project's remotes with the command: \verb|git remote -v|. The remote is listed twice: once for \verb|(fetch)| and once for \verb|(push)|. We'll learn about these later in the lesson

\subsubsection{git fetch}
An easy way to see if changes have been made to the remote and bring the changes down to your local copy is with:\\ \verb|git fetch|

This command will not \textit{merge} changes from the remote into your local repository. It brings those changes onto what's called a \textit{remote branch}. Learn more about how this works below

\subsubsection{git merge}
Even though Sally's new commits have been fetched to your local copy of the Git project, those commits are on the \verb|origin/master| branch. Your \textit{local} \verb|master| branch has not been updated yet, so you can't view or make changes to any of the work she has added

In \textit{Lesson III, Git Branching} we learned how to merge branches. Now we'll use the \verb|git merge| command to integrate \verb|origin/master| into your local \verb|master| branch. The command:\\
\verb|git merge origin/master| \textbar~\verb|git merge origin/dev| will accomplish this for us

\subsubsection{git push}
Now it's time to share our work with Sally.\\
\verb|git push origin your_branch_name| \textbar~\verb|git push origin HEAD:dev|

will push your branch up to the remote, \verb|origin|. From there, Sally can review your branch and merge your work into the master branch, making it part of the definitive project version.



\subsection{create-react-app and git}
\begin{enumerate}
\item \verb|create-react-app myapp| will initialize a git repository

\item create a new repository on github, then push existing repository from command line using following commands\\
\verb|git remote add origin https://github.com/Jiapan-Yu/newTest.git|\\
\verb|git push -u origin master|

\item then create a new branch with \verb|git branch dev| and switch to that branch\\ \verb|git checkout dev|

\end{enumerate}

\subsection{SVN}
\begin{itemize}
\item 项目文件删除了后,如果先update则会恢复删除的文件;如果是文件改动了,如果先update则不会有任何影响。

\item 如果使用 git 来同时管控使用了 svn 的项目,则当 svn 出现 conflicts 的时候,首先用 svn 来解决 conflicts,再 git commit。如果用 vscode 解决了 conflicts,svn 是不认同的,那么我需要再次用 svn 来解决 conflicts 然后再次 git commit。

\end{itemize}

当使用 git 来同时管控使用了 svn 的项目,我是首先 git commit $\Rightarrow$ svn update $\Rightarrow$ svn commit 写上 git commit 的最新 log $\Rightarrow$ git commit 写上 others' changes 的 log,{\color{red}同时}我需要考虑以上两种情况

\subsection{FAQ}
\begin{enumerate}
\item 什么时候需要建立 remote repo?\\
当本地有 master 和 ephemeral 两个分支的时候,我在 master 上面已经改了部分代码,当需要看 ephemeral 分支上面代码作对比的时候,I can't switch branch without stashing current changes on Github Desktop;如果 remote repo 上面有这两个分支,那么可以直接看 ephemeral 分支上面代码 (比如limeng改动了我的代码放在了 ephemeral 分支,我想自己重新在 master 分支上面写一遍同时对比limeng写的)。

如果项目其他成员是用的svn,当某个成员改动了我的代码后,我想 ``hard reset" 后重新写,commit了多次;由于那个成员已经上传到了svn,所以上传svn的时候我需要首先更新本地代码,我得回到原来的状态更新,那么我会失去之前 ``hard reset" 后的 commits。如果有 remote repo,我可以上传 commits,然后退回用 svn 更新本地代码后,复制 remote repo 上面的 commits 文件,对应的写 log (这里面有个问题,就是 remote repo 上面的第一个 commit 由于是基于旧的 svn 上面的代码,所以看到的 changes 是有点不同的,但是后面的 commits 没有影响)({\color{red}commit 指 git commit })

需要 copy 之前写的某个 commit 的一个文件时(包括那个 commit 之前写的内容),github desktop 会有 \verb|@@ -10,7 +11,11 @@| 之类的符号,但是 remote repo 可以没有

\item Your branch and 'origin/master' have diverged, how to undiverge branches?\\
You can review the differences with a: \verb|git log HEAD..origin/master| before pulling it (fetch + merge), see \href{https://stackoverflow.com/questions/2452226/master-branch-and-origin-master-have-diverged-how-to-undiverge-branches}{merge or rebase}

\item How do I list all remote branches in Git 1.7+?
\begin{itemize}
\item \verb|git branch -r| list local copies of remote branches

\item \verb|git ls-remote| list remote branches.\\ \verb|git remote show origin| shows all the branches on the remote, including those that are not tracked locally and even those that have not yet been fetched, it also tries to show the status of the branches relative to your local repo

\item \verb|git remote prune origin| remove info about removed remote branches
\end{itemize}

\item How can I see which git branches are tracking which remote / upstream branch?\\
\verb|git branch -vv   # doubly verbose!|

\item \href{https://stackoverflow.com/questions/89332/how-to-recover-a-dropped-stash-in-git}{How to recover a dropped stash in Git?} (in git bash)
\begin{verbatim}
gitk --all $( git fsck --no-reflog | awk '/dangling commit/ {print $3}' )
\end{verbatim}

\end{enumerate}

\subsection{git rebase}
\verb|git rebase -i[ HEAD~3]| to reorder previous local commits (改动的那个commit之后的commit sha都会改变). Then inside vim:
\begin{enumerate}
\item press uppercase V to select whole line
\item move the cursor to the end of what you want to cut
\item press d to cut (or y to copy)
\item move to where you would like to paste
\item press P to paste before the cursor, or p to paste after
\end{enumerate}

exit vim with \verb|:wq| or quit vim without saving \verb|:cq|

{\color{red}保留一份完整的 commit sha (git log)} 先在 local reset 然后 checkout,then \verb|push --force| 让 remote 恢复到想要的状态。接着 local reset 到最新然后添加新文件(改动),接着使用 \verb|git rebase -i| 调到想要的位置后 push 那一个 commit 到 remote 用 \verb|git push <remotename> <commit SHA>:<remotebranchname>|

\subsection{Modify previous commit messages}
See section ``Amending the message of older or multiple commit messages" of \url{https://help.github.com/en/articles/changing-a-commit-message}

\subsection{vim shortcuts}
\noindent\verb|i| enter insert mode\\
\verb|esc| exit insert mode

\href{https://gist.github.com/awidegreen/3854277}{vim\_cheatsheet}

\subsection{gitignore file}

.gitignore file
\begin{verbatim}
# / is mandatory
/*
!.gitignore
# however !pics/* won't work as expected
# seems like negative exclusion exclamation mark shouldn't combine asterisk
!pics/
!java.tex
!javanote.tex
\end{verbatim}

\subsection{semantic versioning}
向下兼容就是向新的版本兼容,向上就是向老的版本。minor release向下兼容但不向上兼容


\section{VSCode}
\begin{itemize}

\item remove selection next find match: \verb|Cmd+U| to undo the last action

\item debug的时候,开的是一个chrome页面,当需要关闭当前页面重新打开测试的时候该怎么办?

\item source control: modified 为蓝色 \quad added 为暗橙色

\item 搜索只能搜内容,无法搜索哪个文件夹

\item type \verb|tab| key twice to autocomplete

\item VSCode 1.27.2 user settings
\begin{verbatim}
{
    "editor.minimap.enabled": true,
    "emmet.triggerExpansionOnTab": true,
    "window.zoomLevel": 0,
    "editor.tabSize": 2,
    "files.autoSave": "off",
    "explorer.confirmDelete": false,
    "javascript.implicitProjectConfig.experimentalDecorators": true,
    "files.associations": {
        "*.extension": [
            "css",
            "js"
        ]
    },
    "javascript.updateImportsOnFileMove.enabled": "always",
    "debug.toolBarLocation": "hidden",
    "update.channel": "none",
    "window.menuBarVisibility": "default",
    "workbench.statusBar.visible": true,
    "window.titleBarStyle": "custom",
    "terminal.integrated.shell.windows": "C:\\Windows\\System32\\cmd.exe",
    "workbench.editor.enablePreviewFromQuickOpen": false,
    "editor.tabCompletion": true,
    "git.enableSmartCommit": true,
    "git.confirmSync": false,
    "latex-workshop.view.pdf.viewer": "external",
    "editor.parameterHints.enabled": false,
    "editor.suggest.snippetsPreventQuickSuggestions": false,
    "html.suggest.html5": false,
    "editor.snippetSuggestions": "none",
    "diffEditor.ignoreTrimWhitespace": true,
}
\end{verbatim}
\end{itemize}


\section{React}

\subsection{创建React项目 \href{https://react.docschina.org/docs/add-react-to-a-new-app.html}{Add React to a New Application}}
\begin{enumerate}
\item \verb|npm install -g create-react-app|
\item \verb|create-react-app myapp|
\item \verb|cd myapp|
\item \verb|npm start|
\item When you're ready to deploy to production, running \verb|npm run build| will create an optimized build of your app in the \verb|build| folder
\begin{verbatim}
import React from 'react';
import ReactDOM from 'react-dom';

class App extends React.Component {
  render() {
    return (<h1>React setup</h1>);
  }
}

ReactDOM.render(
  <App />,
  document.getElementById('root')
);
\end{verbatim}

\item 默认所有配置是隐藏起来的,要想自定义配置,需要运行一个命令:\verb|npm run eject|。接着package.json里面的scripts键会变为node。\verb|npm start| 运行正常

\item install \verb|less| and \verb|mini-css-extract-plugin| to use less and extract css into one file, configured \verb|webpack.config.dev.js|. Add following lines in the correct places
\begin{verbatim}
const MiniCssExtractPlugin = require("mini-css-extract-plugin");

,
// adds support for css and less
{
  test: /\.(css|less)$/, /* need this less */
  use: [
    MiniCssExtractPlugin.loader,
    "css-loader",
    "less-loader" /* need this line */
  ]
}

new MiniCssExtractPlugin({
  // Options similar to the same options in webpackOptions.output
  // both options are optional
  filename: "[name].css",
  chunkFilename: "[id].css"
}),
\end{verbatim}




\end{enumerate}



\subsection{JSX}
\begin{enumerate}

\item \textbf{Event Listeners in JSX}
\begin{verbatim}
<img onClick={myFunc} />
\end{verbatim}
An event listener attribute's name should be something like \verb|onClick| or \verb|onMouseOver|: the word \verb|on|, plus the type of event that you're listening for. \href{https://reactjs.org/docs/events.html#supported-events}{Supported Events}

\textbf{Note} that in HTML, event listener names are written in all lowercase, such as \verb|onclick| or \verb|onmouseover|. In JSX, event listener names are written in camelCase, such as \verb|onClick| or \verb|onMouseOver|

\item \textbf{Variable Attributes in JSX}
\begin{verbatim}
// Use a variable to set the `height` and `width` attributes:

const sideLength = "200px";

const panda = (
  <img 
    src="images/panda.jpg" 
    alt="panda" 
    height={sideLength} 
    width={sideLength} />
);
\end{verbatim}

\item Everything inside of the curly braces will be treated as regular JavaScript
\begin{verbatim}
ReactDOM.render(
  <h1>{2 + 3}</h1>,
  document.getElementById('app')
);
\end{verbatim}
The result is 5

\item \verb|ReactDOM.render()| is the most common way to render JSX:
\begin{verbatim}
ReactDOM.render(<h1>Hello world</h1>, document.getElementById('app'));
\end{verbatim}

\item There's a rule that we haven't mentioned: a JSX expression must have exactly one outermost element.

\item If a JSX expression takes up more than one line, then you must wrap the multi-line JSX expression in parentheses. This looks strange at first, but you get used to it:
\begin{verbatim}
const theExample = (
   <a href="https://www.example.com">
     <h1>
       Click me!
     </h1>
   </a>
 );
\end{verbatim}

\item JSX elements are treated as JavaScript expressions. They can go anywhere that JavaScript expressions can go.

That means that a JSX element can be saved in a variable, passed to a function, stored in an object or array... you name it

\item JSX is a syntax extension for JavaScript. It was written to be used with React. JSX code looks a lot like HTML
\end{enumerate}

\subsubsection{JSX caveats}
\begin{enumerate}

\item You can't use JSX until you've imported react

\item One outermost tag

\item You can not inject an if statement into a JSX expression.

This code will break:
\begin{verbatim}
(
  <h1>
    {
      if (purchase.complete) {
        'Thank you for placing an order!'
      }
    }
  </h1>
)
\end{verbatim}

\item In JSX, self-closing tags have to include the slash, otherwise it will raise an error

\item In JSX, you can't use the word class! You have to use className instead:
\begin{verbatim}
<h1 className="big">Hey</h1>
\end{verbatim}
This is because JSX gets translated into JavaScript, and \verb|class| is a reserved word in JavaScript

\end{enumerate}

\subsubsection{JSX conditionals}
\begin{enumerate}
\item The Ternary Operator
\begin{verbatim}
const headline = (
  <h1>
    { age >= drinkingAge ? 'Buy Drink' : 'Do Teen Stuff' }
  </h1>
);
\end{verbatim}
\item \textbf{\&\&}
\begin{verbatim}
const tasty = (
  <ul>
    <li>Applesauce</li>
    { !baby && <li>Pizza</li> }
    { age > 15 && <li>Brussels Sprouts</li> }
    { age > 20 && <li>Oysters</li> }
    { age > 25 && <li>Grappa</li> }
  </ul>
);
\end{verbatim}
\end{enumerate}

\subsubsection{.map in JSX}
If you want to create a list of JSX elements, then .map() is often your best bet. It can look odd at first:
\begin{verbatim}
const strings = ['Home', 'Shop', 'About Me'];

const listItems = strings.map(string => <li>{string}</li>);

<ul>{listItems}</ul>
\end{verbatim}

\subsubsection{Keys in JSX}
A \verb|key| is a JSX attribute. The attribute's name is \verb|key|. The attribute's value should be something unique, similar to an \verb|id| attribute
\begin{verbatim}
const people = ['Rowe', 'Prevost', 'Gare'];

const peopleLis = people.map((person, i) =>
  // expression goes here:
  <li key={'person_' + i}>{person}</li>
);
\end{verbatim}

\subsection{The component}
\begin{enumerate}
\item React applications are made out of \textit{components}. A component is a small, reusable chunk of code that is responsible for one job. That job is often to render some HTML.
\begin{verbatim}
import React from 'react';
import ReactDOM from 'react-dom';

class MyComponentClass extends React.Component {
  render() {
    return <h1>Hello world</h1>;
  }
};

ReactDOM.render(
  <MyComponentClass />,
  document.getElementById('app')
);
\end{verbatim}

\item For now, just know that you get the React library via \verb|import React from 'react';|

\item To clarify: the DOM is \textit{used} in React applications, but it isn't \textit{part} of React\\
\verb|import ReactDOM from 'react-dom';|

\item By subclassing \verb|React.Component|, you create a new component class. This is not a component! A component class is more like a factory that produces components. When you start making components, each one will come from a component class

\item \textbf{The Render Function:} All you know so far is that its name is \verb|render|, it needs a return statement for some reason, and you have to include it in the body of your component class declaration

\item \textbf{Create a Component Instance: }\\\verb|<MyComponentClass />|

\item \textbf{Render A Component: }
\verb|<MyComponentClass />| will call its render method, which will return the JSX element \verb|<h1>Hello world</h1>|. \verb|ReactDOM.render()| will then take that resulting JSX element, and add it to the virtual DOM. This will make "Hello world" appear on the screen.

\end{enumerate}

\subsection{Components and advanced JSX}
\subsubsection{Put Logic in a Render Function}
\begin{verbatim}
class Random extends React.Component {
  render() {   
    const n = Math.floor(Math.random() * 10 + 1);    
    return <h1>The number is {n}!</h1>;
  }
}
\end{verbatim}

\subsubsection{Use this in a Component}
\begin{verbatim}
class IceCreamGuy extends React.Component {
  get food() {
    return 'ice cream';
  }

  render() {
    return <h1>I like {this.food}.</h1>;
  }
}
\end{verbatim}

\subsubsection{Use an Event Listener in a Component}
\begin{verbatim}
render() {
  return (
    <div onHover={myFunc}>
    </div>
  );
}
\end{verbatim}
Recall that an event handler is a function that gets called in response to an event. In the above example, the event handler is \verb|myFunc()|.

In React, you define event handlers (functions) as methods on a component class. Like this:
\begin{verbatim}
class MyClass extends React.Component {
  myFunc() {
    alert('Stop it.  Stop hovering.');
  }

  render() {
    return (
      <div onHover={this.myFunc}>
      </div>
    );
  }
}
\end{verbatim}
Almost all functions that you define in React will be defined in this way, as methods in a class


\subsection{Components render other components}
\subsubsection{A Component in a Render Function}
\begin{enumerate}

\item When you use named exports, you always need to wrap your imported names in curly braces, such as:\\
\verb|import { faveManifestos, alsoRan } from './Manifestos';|

\item To import a variable, you can use an \verb|import| statement:\\
\verb|import { NavBar } from './NavBar.js';|\\
you can omit the \verb|.js| extension

\item Render methods can also return another kind of JSX: component instances
\begin{verbatim}
class OMG extends React.Component {
  render() {
    return <h1>Whooaa!</h1>;
  }
}

class Crazy extends React.Component {
  render() {
    return <OMG />;
  }
}
\end{verbatim}
In the above example, \verb|Crazy's| render method \verb|returns| an instance of the \verb|OMG| component class. You could say that \verb|Crazy| renders an \verb|<OMG />|
\end{enumerate}

\subsection{this.props}
Information that gets passed from one component to another is known as "props."

\subsubsection{Access a Component's props}
To see a component's props object, you use the expression \verb|this.props|
\begin{verbatim}
class PropsDisplayer extends React.Component {
  render() {
  	const stringProps = JSON.stringify(this.props);

    return (
      <div>
        <h1>CHECK OUT MY PROPS OBJECT</h1>
        <h2>{stringProps}</h2>
      </div>
    );
  }
}

ReactDOM.render(<PropsDisplayer />, document.getElementById('app'));
\end{verbatim}

\subsubsection{Pass `props` to a Component}
If you want to pass information that isn't a string, then wrap that information in curly braces\\
\verb|<Greeting myInfo={["top", "secret", "lol"]} />|\\
\verb|<Greeting name="Frarthur" town="Flundon" age={2} haunted={false} />|

\subsubsection{Render a Component's props}
\begin{verbatim}
class Greeting extends React.Component {
  render() {
    return <h1>Hi there, {this.props.firstName}!</h1>;
  }
}

ReactDOM.render(
  <Greeting firstName='Joshua' />, 
  document.getElementById('app')
);
\end{verbatim}

\subsubsection{Pass props From Component To Component}
\textbf{A curmudgeonly clarification about grammar:} You may have noticed some loose usage of the words prop and \verb|props|. \verb|props| is the name of the object that stores passed-in information. \verb|this.props| refers to that storage object. At the same time, each piece of passed-in information is called a prop. This means that \verb|props| could refer to two pieces of passed-in information, or it could refer to the object that stores those pieces of information

\textbf{Greeting.js}
\begin{verbatim}
import React from 'react';

export class Greeting extends React.Component {
  render() {
    return <h1>Hi there, {this.props.name}!</h1>;
  }
}
\end{verbatim}

\textbf{App.js}
\begin{verbatim}
import React from 'react';
import ReactDOM from 'react-dom';
import {Greeting} from './Greeting';

class App extends React.Component {
  render() {
    return (
      <div>
        <h1>
          Hullo and, "Welcome to The Newzz," "On Line!"
        </h1>
        <Greeting name="Joshua" />
        <article>
          Latest newzz:  where is my phone?
        </article>
      </div>
    );
  }
}

ReactDOM.render(
  <App />, 
  document.getElementById('app')
);
\end{verbatim}

\subsubsection{Receive an Event Handler as a prop}
\textbf{Button.js}
\begin{verbatim}
import React from 'react';

export class Button extends React.Component {
  render() {
    return (
      <button onClick={this.props.onClick}>
        Click me!
      </button>
    );
  }
}
\end{verbatim}

\textbf{Talker.js}
\begin{verbatim}
import React from 'react';
import ReactDOM from 'react-dom';
import { Button } from './Button';

class Talker extends React.Component {
  handleClick() {
    let speech = '';
    for (let i = 0; i < 10000; i++) {
      speech += 'blah ';
    }
    alert(speech);
  }
  
  render() {
    return <Button onClick={this.handleClick} />;
  }
}

ReactDOM.render(
  <Talker />,
  document.getElementById('app')
);
\end{verbatim}

Great! You just passed a function from \verb|<Talker />| to \verb|<Button />|. \verb|<Talker />| is the \textit{parent} component class.

\textbf{Name confusion demystify: }\verb|<Button />| is not an HTML-like JSX element; it's a component instance. Names like onClick only create event listeners if they're used on HTML-like JSX elements. Otherwise, they're just ordinary prop names.

\subsubsection{this.props.children}
\begin{verbatim}
import { LilButton } from './LilButton';

class BigButton extends React.Component {
  render() {
    console.log(this.props.children);
    return <button>Yo I am big</button>;
  }
}


// Example 1
<BigButton>
  I am a child of BigButton.
</BigButton>


// Example 2
<BigButton>
  <LilButton />
</BigButton>


// Example 3
<BigButton />
\end{verbatim}

In Example 1, \verb|<BigButton>|'s \verb|this.props.children| would equal the text, "I am a child of BigButton."

In Example 2, \verb|<BigButton>|'s \verb|this.props.children| would equal a \verb|<LilButton />| component.

In Example 3, \verb|<BigButton>|'s \verb|this.props.children| would equal \verb|undefined|.

If a component has more than one child between its JSX tags, then \verb|this.props.children| will return those children in an array. However, if a component has only one child, then \verb|this.props.children| will return the single child, not wrapped in an array.

\subsubsection{defaultProps}
\begin{verbatim}
class Button extends React.Component {
  render() {
    return (
      <button>
        {this.props.text}
      </button>
    );
  }
}

// defaultProps goes here:
Button.defaultProps = { text: 'I am a button' };

ReactDOM.render(
  <Button />, 
  document.getElementById('app')
);
\end{verbatim}

\subsection{this.state}
Dynamic information is information that can change. There are two ways for a component to get dynamic information: \verb|props| and \verb|state|. Besides \verb|props| and \verb|state|, every value used in a component should always stay exactly the same

\subsubsection{Setting Initial State}
To make a component have \verb|state|, give the component a \verb|state| property. This property should be declared inside of a constructor method, like this
\begin{verbatim}
class Example extends React.Component {
  constructor(props) {
    super(props);
    this.state = { mood: 'decent' };
  }

  render() {
    return <div></div>;
  }
}

<Example />
\end{verbatim}
\verb|this.state| should be equal to an object, like in the example above. This object represents the initial "state" of any component instance

\subsubsection{Update state with this.setState}
\verb|this.setState()| takes two arguments: an object that will update the component's state, and a callback. You basically never need the callback (the callback is executed after \verb|render()| function)

\subsubsection{Call this.setState from Another Function}
\begin{verbatim}
class Mood extends React.Component {
  constructor(props) {
    super(props);
    this.state = { mood: 'good' };
    this.toggleMood = this.toggleMood.bind(this);
  }

  toggleMood() {
    const newMood = this.state.mood == 'good' ? 'bad' : 'good';
    this.setState({ mood: newMood });
  }

  render() {
    return (
      <div>
        <h1>I'm feeling {this.state.mood}!</h1>
        <button onClick={this.toggleMood}>
          Click Me
        </button>
      </div>
    );
  }
}
\end{verbatim}
in React, whenever you define an event handler that uses \verb|this|, you need to add \verb|this.methodName = this.methodName.bind(this)| to your constructor function.

\textit{Any time that you call this.setState(), this.setState() AUTOMATICALLY calls .render() as soon as the state has changed}.

Think of \verb|this.setState()| as actually being two things: \verb|this.setState()|, immediately followed by \verb|.render()|. Set breakpoint in \verb|render()| method in order to inspect the changes made by \verb|this.setState()| in chrome

That is why you can't call \verb|this.setState()| from inside of the \verb|.render()| method! \verb|this.setState()| \textit{automatically} calls \verb|.render()|. If \verb|.render()| calls \verb|this.setState()|, then an infinite loop is created

\subsection{Child Components Update Their Parents' state}
See javascript.pdf

\subsection{Child Components Update Their Siblings' props}
See codecademy, this pattern occurs in React all the time!


\subsection{A controlled component example}
\begin{verbatim}
class NameForm extends React.Component {
  constructor(props) {
    super(props);
    this.state = {value: ''};

    this.handleChange = this.handleChange.bind(this);
    this.handleSubmit = this.handleSubmit.bind(this);
  }

  handleChange(event) {
    this.setState({value: event.target.value});
  }

  handleSubmit(event) {
    alert('A name was submitted: ' + this.state.value);
    event.preventDefault();
  }

  render() {
    return (
      <form onSubmit={this.handleSubmit}>
        <label>
          Name:
          <input type="text" value={this.state.value} onChange={this.handleChange} />
        </label>
        <input type="submit" value="Submit" />
      </form>
    );
  }
}

ReactDOM.render(
  <NameForm />,
  document.getElementById('root')
);
\end{verbatim}


\subsection{Advanced React Techniques}

\subsubsection{React styles}
In regular JavaScript, style names are written in hyphenated-lowercase:
\begin{verbatim}
const styles = {
  'margin-top':       "20px",
  'background-color': "green"
};
\end{verbatim}

In React, those same names are instead written in camelCase:
\begin{verbatim}
const styles = {
  marginTop:       20,
  backgroundColor: "green"
};
\end{verbatim}
This has zero effect on style property values, only on style property names.

In React, if you write a style value as a number {\color{red}(meaning no quotation marks)}, then the unit \verb|"px"| is assumed. The exception is \verb|lineHeight| attribute, because it accepts number and length at the same time

If you want to use units other than "px," you can use a string: \verb|{ fontSize: "2em" }|



\subsubsection{Another React pattern}
A presentational component will always get rendered by a container component

When you separate a container component from a presentational component, the \textit{presentational} component will always end up like this: one \verb|render()| function, and no other properties
\begin{verbatim}
import React from 'react';

export class GuineaPigs extends React.Component {
  render() {
    let src = this.props.src;
    return (
      <div>
        <h1>Cute Guinea Pigs</h1>
        <img src={src} />
      </div>
    );
  }
}
\end{verbatim}
If you have a component class with nothing but a render function, then you can rewrite that component class in a very different way. Instead of using React.Component, you can write it as JavaScript function! A component class written as a function is called a \textit{stateless functional component}\\
See javascript.pdf for an example

\subsubsection{propTypes}
\begin{verbatim}
import React from 'react';

export class MessageDisplayer extends React.Component {
  render() {
    return <h1>{this.props.message}</h1>;
  }
}

// This propTypes object should have
// one property for each expected prop:
MessageDisplayer.propTypes = {
  message: React.PropTypes.string
};
\end{verbatim}
Notice that the value of propTypes is an object, not a function! Each property on the \verb|propTypes| object is called a \verb|propType|

\begin{verbatim}
Runner.propTypes = {
  message:   React.PropTypes.string.isRequired,
  style:     React.PropTypes.object.isRequired,
  isMetric:  React.PropTypes.bool.isRequired,
  miles:     React.PropTypes.number.isRequired,
  milesToKM: React.PropTypes.func.isRequired,
  races:     React.PropTypes.array.isRequired
};
\end{verbatim}
\verb|Runner| has six \verb|propTypes|! Look at each one. Note that \verb|bool| and \verb|func| are abbreviated, but all other datatypes are spelled normally.

\subsubsection{PropTypes in Stateless Functional Components}
\begin{verbatim}
import React from 'react';

export const GuineaPigs = (props) => {  
  let src = props.src;
  return (
    <div>
      <h1>Cute Guinea Pigs</h1>
      <img src={src} />
    </div>
  );  
}

GuineaPigs.propTypes = {
  src: React.PropTypes.string.isRequired
};
\end{verbatim}

\subsubsection{React forms}


\subsection{Lifecycle Methods}
\subsubsection{Mounting lifecycle methods}
\verb|componentWillMount(){} render(){} componentDidMount(){}|

When a component renders for the first time, \verb|componentWillMount| gets called right before \verb|render|.

When a component renders for the first time, \verb|componentDidMount| gets called right after the HTML from \verb|render| has finished loading.

Mounting lifecycle events only execute the first time that a component renders.

If your React app uses AJAX to fetch initial data from an API, then \verb|componentDidMount| is the place to make that AJAX call. More generally, \verb|componentDidMount| is a good place to connect a React app to external applications, such as web APIs or JavaScript frameworks. \verb|componentDidMount| is also the place to set timers using \verb|setTimeout| or \verb|setInterval|.


\subsubsection{Updating/unmounting lifecycle methods}
There are five updating lifecycle methods:
\begin{itemize}
\item \verb|componentWillReceiveProps|
\item \verb|shouldComponentUpdate|
\item \verb|componentWillUpdate|
\item \verb|render|
\item \verb|componentDidUpdate|
\end{itemize}
Whenever a component instance updates, it automatically calls all five of these methods, in order.

\verb|componentWillUnmount| is the only unmounting lifecycle method.

\verb|componentWillUnmount| gets called right before a component is removed from the DOM. If a component initiates any methods that require cleanup, then \verb|componentWillUnmount| is where you should put that cleanup


\section{react}
\begin{enumerate}
\item 
\begin{itemize}
\item ES6 React 组件引用本地图片问题: require里只能写字符串,不能写变量,可以 \verb|require(`${imgSrc}`)|, see \href{https://stackoverflow.com/questions/39999367/how-do-i-reference-a-local-image-in-react}{How do I reference a local image in React?}

\verb|document.getElementById('progress-col').firstElementChild|\newline
\verb|.setAttribute("style", `background-image:|\newline
\verb| url(${require("../../static/img/rectBgHighlighted.svg")});`);|

或者

\verb|document.getElementById('progress-col').firstElementChild.style|\newline
\verb|.backgroundImage = `url(${require("../../static/img/rectBgHighlighted.svg")})`;|


\item 当引用视频文件时
\begin{verbatim}
import covervid from './movie.mp4';
<video controls style={{ width: "71rem", height: "37rem", }}>
  <source src={covervid} type="video/mp4" />
</video>
\end{verbatim}
或者
\begin{verbatim}
<video controls style={{ width: "71rem", height: "37rem", }}>
  <source src={require("./movie.mp4")} type="video/mp4" />
</video>
\end{verbatim}

\vspace{1in}
\begin{verbatim}
// videoSrc = http://127.0.0.1:8081/data/limeng/Test/002/Raw/FE/
// ColorImg/test.mp4
const { videoSrc } = this.state;

<video controls style={{ width: "71rem", height: "37rem", }}>
  {
    // <source src={videoSrc} type="video/mp4" /> // 为什么不行?
    videoSrc && <source src={videoSrc} type="video/mp4" />
  }
</video>
\end{verbatim}
这样写也有问题,当 \verb|videoSrc| 为空字符串时,可能由于视频缓存问题视频仍然会展示并且可以播放。更准确的写法应该是
\begin{verbatim}
<div>
  {
    videoSrc &&
    <video controls style={{ width: "71rem", height: "37rem", }}>
      <source src={videoSrc} type="video/mp4" />
    </video>
  }
</div>
\end{verbatim}
或者
\begin{verbatim}
<div>
  {
    videoSrc &&
    <video controls src={videoSrc} style={{ width: "71rem", height: "37rem", }}>
    </video>
  }
</div>
\end{verbatim}

\end{itemize}

\item 

\end{enumerate}

\subsection{Caveats}
\begin{enumerate}
\item dva: services, models, pages, utils, components, config (包括router.config.js)

\item 如果数据不展示或者展示不正确,需要从数据源处找错误

\item 刷新不展示 $\rightarrow$ 需要看路由、看父页面

\item 如果请求失败则redirect到登录页面 $\rightarrow$ 在一个总组件(eg. BasicLayout.js 每个请求都会走的组件),如果后台response的success为false,则router.push('/login') {\color{red}此方法不正确,应该到请求拦截的地方处理}

\item 登出清缓存 $\rightarrow$ logout方法里 \verb|localStorage.clear()|

\item 关闭当前标签或浏览器时清缓存 $\rightarrow$ 在MDN上查beforeunload监听事件,然后在回调方法里面清\verb|localStorage| 

\item 对js的方法(特别是数组、字符串的方法)不熟,对antd标签的属性不熟

\item treeSelect属性数据对应的时候,一些需要注意的地方

\end{enumerate}

\subsection{JSX Spread Attributes}
The specification order is important. Later attributes override previous ones.
\begin{verbatim}
var props = { foo: 'default' };
var component = <Component {...props} foo={'override'} />;
console.log(component.props.foo); // 'override'
\end{verbatim}


\subsection{this.setState}
\href{https://github.com/yannickcr/eslint-plugin-react/blob/HEAD/docs/rules/no-access-state-in-setstate.md}{Prevent using this.state within a this.setState (react/no-access-state-in-setstate)}

This can be avoided with using callbacks which takes the previous state as first argument:
\begin{verbatim}
state = {
  value: 1,
};

increment = () => {
  this.setState(prevState => ({
    value: prevState.value + 1,
  }));
}

this.increment();
this.increment();
\end{verbatim}

\subsection{react this {\color{red}以下都为没在constructor里面进行this邦定}}
可以用普通函数的地方都可以用箭头函数。另外箭头函数在某些情况下 (分是否只传event还是有其它参数和是在class methods处还是在事件触发处) 可以替代bind

在React里面,传参要看传的只是event{\color{red}还是含有} (event貌似是一定有的) 其它参数 (包括其它参数为空的情形)。当含有其它参数的时候,有以下几种写法 (可以参考ylx-pc门票列表),箭头函数当用于事件触发的地方时能替代bind,当class methods使用的时候就{\color{red}不}可以替代bind
\begin{enumerate}
\item \begin{verbatim}
handleOnClick(item) {
  this.props.handleOnClick(item)
}
或
handleOnClick = (item) => {
  this.props.handleOnClick(item)
}

<div className="history" onClick={() => this.handleOnClick(item)}></div>
\end{verbatim}

\item \begin{verbatim}
handleOnClick = (item) => {
  this.props.handleOnClick(item)
}

<div className="history" onClick={this.handleOnClick.bind(this, item)}></div>
下面一行代码不行
<div className="history" onClick={this.handleOnClick(item)}></div>
\end{verbatim}

\item \begin{verbatim}
handleOnClick(item) {
  this.props.handleOnClick(item)
}

<div className="history" onClick={this.handleOnClick.bind(this, item)}></div>
\end{verbatim}

\item \begin{verbatim}
handleOnClick = (item) => () => {
  this.props.handleOnClick(item)
}  //called currying

<div className="history" onClick={this.handleOnClick(item)}></div>
下面一行代码不行
<div className="history" onClick={this.handleOnClick.bind(this, item)}></div>
\end{verbatim}
\end{enumerate}

当只为event的时候,箭头函数当class methods使用的时候可以替代bind (见1)。 但是当用于事件触发的地方时就不能替代bind (见2) (可以参考javascript.pdf Child Components Update Their Parents' state in React)。可以用普通函数的地方都可以用箭头函数 (见3)
\begin{enumerate}
\item \begin{verbatim}
handleChange = (e) => {
  const name = e.target.value;
  this.props.onChange(name);
} // 必须用箭头函数,普通函数不行

<select id="great-names" onChange={this.handleChange}></select>
\end{verbatim}

\item \begin{verbatim}
handleChange(e) {
  const name = e.target.value;
  this.props.onChange(name);
}

<select id="great-names" onChange={this.handleChange.bind(this)}></select>
下面一行代码不行
<select id="great-names" onChange={() => this.handleChange}></select>
\end{verbatim}

\item \begin{verbatim}
handleChange = (e) => {
  const name = e.target.value;
  this.props.onChange(name);
}

<select id="great-names" onChange={this.handleChange.bind(this)}></select>
下面一行代码不行
<select id="great-names" onChange={() => this.handleChange}></select>
\end{verbatim}

\end{enumerate}

\includegraphics[width=6.5in]{./childUpdateParentsState.png}\\
The body of handleChange won't be executed, don't know why. 当传的是其它参数的时候 (图片中的情形), 在事件触发的地方调用class method需要加(),否则对应的class method的body不会被执行


\subsection{react styles}
在项目中引用antd某个组件时,如果想对其添加style,则需要将那个组件包裹在div内,接着在div上加style属性,不能直接在那个组件上加style属性 (但是可以找到对应的className在less文件里面加样式)\\
\includegraphics{./pics/styleOnComponent.png}


\subsection{less}
\href{https://medium.com/@joseph0crick/react-css-modules-less-webpack-4-a50d902d0a3}{React + CSS Modules + LESS + Webpack 4}
包含一个完整的less示例

\subsection{Array map method}
\begin{enumerate}
\item 

\item Each child in an array or iterator should have a unique "key" prop

\end{enumerate}

\subsection{react高阶组件 (high order component)}
\begin{itemize}
\item 高阶组件就是接受一个组件作为参数并返回一个新组件的函数

\item 高阶组件是一个函数,并不是组件

\item 尽量使用代理方式的高阶组件

\item 

\end{itemize}

\subsection{react项目js文件import的顺序}
首先 import 与 react 相关的库比如 \verb|react|、\verb|react-dom| 等,再 import 本地文件和组件,接下来 import 其它库比如 \verb|moment|、\verb|antd| 等,接下来 import 接口,最后 import 样式文件。


\section{react-redux}
react-redux使用方法:
\begin{enumerate}
\item 提交 dispatch 触发 actions

\item 应该是有一个 store 目录

\item 在需要使用 store 中数据的组件中,引入 react-redux 提供的 connect 方法\\ \verb|import { connect } from 'react-redux'|,接着,在抛出的时候调用 connect 方法改造当前组件 \verb|export default connect(state => state)(SelectItem)|

\item 然后,我们在被改造的组件内就可以通过this.props.属性名获取store中的数据了,我在 'combineReducers' 中传的是一个以 data 为键的对象,所以我在当前组件中使用 this.props.data

\end{enumerate}


\section{MobX}
store在刷新后数据会被清掉,页面跳转数据会保留


\section{Vue.js}
\begin{enumerate}
\item We call constantly-changing data values like this \textbf{dynamic data}

\item Vue allows us to store data that can be calculated using values from the \verb|data| object at a separate property called \verb|computed|. dynamic data based on other dynamic data is best stored as a \verb|computed| property

\item \verb|v-bind:disabled| $\Rightarrow$ \verb|:disabled| \quad \verb|v-on:click| $\Rightarrow$ \verb|@click|

\item The value of \verb|watch| is an object containing all of the properties to watch. The keys of this object are the names of the properties to watch for changes and the values are functions to run whenever the corresponding properties change. These functions take two parameters: the new value of that property and the previous value of that property

\end{enumerate}

\subsection{Templates}
\begin{enumerate}
\item A place to store the data we will be displaying
\item A syntax for displaying that information
\end{enumerate}

\begin{verbatim}
<div id="app">
  <h2>Hello, {{ username }}</h2>
</div>
\end{verbatim}

In this example, \verb|{{ username }}| will be filled in with the value of \verb|username| from the Vue app's \verb|.data| object when the page is rendered to the user. If the value of \verb|username| changes, the value displayed to the user will be changed as well

Whenever you want to display information from the Vue app's \verb|data|, you wrap the name of the \verb|.data| property in two sets of mustaches (curly brackets) and the expression will be replaced with the Vue data information for the end user to see.

\subsection{Instance Methods}
there is an options object property called methods
\begin{verbatim}
const app = new Vue({
  el: "#app",
  data: {
    hoursStudied: 300
  },
  methods: {
    resetProgress: function () {
      this.hoursStudied = 0;
    }
  }
});

<button v-on:click="resetProgress">Reset Progress</button>
\end{verbatim}

\subsection{Text, Textarea, and Select Bindings}
\begin{verbatim}
<input type="text" v-model="username" />

const app = new Vue({ 
  el: '#app',
  data: { username: 'Michael' } 
});
\end{verbatim}
Now, when this example site is loaded, the \verb|<input>| will already be pre-filled with \verb|'Michael'|, the starting value of \verb|username|. Then, whenever the \verb|<input>| is modified by the user, the \verb|username| \verb|data| value will automatically change to the value typed in by the user.

\verb|v-model| works on all HTML form field elements. So, simple form fields such as \verb|<textarea>| elements and \verb|<select>| elements can be bound to \verb|data| and \verb|computed| properties in the exact same way: adding \verb|v-model="propertyName"| to the opening tag of the elements..

\subsection{Vue Forms}
\begin{itemize}
\item Form fields can be bound to Vue data using the \verb|v-model| directive — how \verb|v-model| is used depends on the type of field it is being added to

\item Form event handlers can be added using \verb|v-on:submit| and \verb|v-on:reset|

\item Modifiers can be used to add functionality to directives — most importantly preventing page reload on form submission using \verb|v-on:submit.prevent| and cleaning up form field values using \verb|.number| and \verb|.trim|

\item Form validation can be implemented by setting the value of the \verb|disabled| attribute on a \verb|<button>| to the value of a computed property using \verb|v-bind|
\end{itemize}

\subsection{Classes}
\begin{verbatim}
<span v-bind:class="{ unread: hasNotifications }">Notifications</span>

.unread {
  background-color: blue;
}

const app = new Vue({
  data: { notifications: [ ... ] },
  computed: {
    hasNotifications: function() {
      return notifications.length > 0;
    }
  }
}
\end{verbatim}
In this example, we are using the \verb|v-bind:class| directive to dynamically add a class called \verb|unread| to a ``Notifications" \verb|<span>| element if the \verb|computed| property \verb|hasNotifications| returns \verb|true|.

\verb|v-bind:class| takes an object as its value — the keys of this object are class names and the values are Vue app properties that return a truthy or falsy value. If the value is truthy, the class will be added to the element — otherwise it will not be added.

\subsection{Class Arrays}
\begin{verbatim}
<span v-bind:class="[{ unread: hasNotifications }, menuItemClass]">Notifications</span>

const app = new Vue({
  data: { 
    notifications: [ ... ],
    menuItemClass: 'menu-item'
  },
  computed: {
    hasNotifications: function() {
      return notifications.length > 0;
    }
  }
}

.menu-item {
  font-size: 12px;
}

.unread {
  background-color: blue;
}
\end{verbatim}
The object at the beginning of the array will still conditionally add the \verb|unread| class based on whether there are unread notifications. However, we now always add the class stored at \verb|menuItemClass|, \verb|menu-item|, to our "Notifications" element.

\subsection{Review}
We learned how to dynamically add inline styles using \verb|v-bind:style| with a style object or an array of style objects. We then learned how to dynamically add classes using \verb|v-bind:class| with a class object or an array of class objects and class name strings.


\section{JavaScript}

\subsection{Notes}
\begin{enumerate}
\item 注意methods和events调用时候的区别

\item the error stops execution in a program

\item session survives over page reloads and restores

\item 当后台服务重启后,可在请求统一响应拦截的地方根据TypeError来跳到登录页

\item 报错了主要看 chrome console 里面

\end{enumerate}

\subsection{Differences between expressions, declarations and statements}
An expression produces a value and can be written wherever a value is expected. Each of the following lines contains an expression:
\begin{verbatim}
myvar
3 + x
myfunc("a", "b")
\end{verbatim}

Wherever JavaScript expects a statement, you can also write an expression (in Java, you can't). Such a statement is called an \textit{expression statement}.

A statement is a piece of code that tells the computer to do something. Here are the most common types of statements:
\begin{verbatim}
var i;                        // variable declaration
i = 5;                        // value assignment
i = i + 1;                    // value assignment
i++;                          // same as above
var x = 9;                    // declaration & assignment
var fun = function() {...};   // var decl., assignmt, and func. defin.
alert("hi");                  // function call
\end{verbatim}
All of these statements \textbf{can} end with a \verb|;| but none of them \textbf{must}. The semicolon is only obligatory when you have two or more statements \textbf{on the same line}:
\begin{verbatim}
var i = 0; i++        // <-- semicolon obligatory
                      //     (but optional before newline)
var i = 0             // <-- semicolon optional
    i++               // <-- semicolon optional
\end{verbatim}
You \textbf{shouldn't} put a semicolon after a closing curly bracket \verb|}|. The only exceptions are \textbf{assignment statements}, such as \verb|var obj = {};|, see above.


\subsection{\href{https://www.cnblogs.com/wangyunhui/p/7511615.html}{js扩展运算符(spread)是三个点(...)}}
作用:将一个数组转为用逗号分隔的参数序列

\subsection{array}

\subsubsection{Arrays with let and const (from codecademy)}
Variables declared with the \verb|const| keyword cannot be reassigned. However, elements in an array declared with \verb|const| remain mutable. Meaning that we can change the contents of a \verb|const| array, but cannot reassign a new array or a different value.

\subsubsection{Array-Like Objects \href{https://dzone.com/articles/js-array-from-an-array-like-object}{JS Array From an Array-Like Object}}
Some objects in JavaScript look like an array, but they aren't one. That usually means that they have indexed access and a length property, but none of the array methods. Examples include the special variable arguments, DOM node lists, and strings.

The (special variable) \verb|arguments| inside a function is an example of an `array-like object'
\begin{verbatim}
function list() {
  return Array.prototype.slice.call(arguments);
}

var list1 = list(1, 2, 3); // [1, 2, 3]
\end{verbatim}

Following method is better for converting variable arguments to an array
\begin{verbatim}
function list(...args) {
  return args
}

let list1 = list(1, 2, 3); // [1, 2, 3]
console.log(list1);
\end{verbatim}

\subsubsection{Array methods}
Array has ``mutator methods", ``accessor methods" and ``iteration methods".

\begin{itemize}
\item whether modify the original array (eg. push())

\item whether return (eg. forEach() doesn't return)

\item return value
\begin{enumerate}
\item whether return an array (eg. reduce() can return any type of your choice)

\item the returned array whether has the original array's size

\item returned array whether retain the element or the return value (eg. filter() retain the element, not the return value)

\item returned array whether has the same value as the original array
\end{enumerate}

\end{itemize}

\subsection{build-in iterable objects}
\href{https://developer.mozilla.org/en-US/docs/Web/JavaScript/Reference/Iteration_protocols#Built-in_iterables}{Built-in\_iterables}

\subsection{object}
\begin{enumerate}
\item There are only seven fundamental data types in JavaScript, and six of those are the primitive data types: \verb|string, number, boolean, null, undefined|, and \verb|symbol|. With the seventh type, \verb|objects|, we open our code to more complex possibilities

\item Objects are \textit{mutable} meaning we can update them after we create them! We can change their properties even when they're declared with const

\item You can delete a property from an object with the \verb|delete| operator
\begin{verbatim}
const spaceship = {
  'Fuel Type': 'Turbo Fuel',
  homePlanet: 'Earth',
  mission: 'Explore the universe' 
};

delete spaceship['Fuel Type'];
\end{verbatim}

\item When the data stored on an object is a function we call that a \textit{method}

\item Certain languages have privacy built-in for objects, but JavaScript does not have this feature. Rather, JavaScript developers follow naming conventions that signal to other developers how to interact with a property

\item \verb|typeof()|: in JavaScript, the data type of null is an object. The return value of \verb|typeof()| can be "string" "number" "boolean" "undefined" "object" "function"

\verb|typeof [1, 2, 4]| is 'object'; in chrome, the \_\_proto\_\_ of an array is an array rather than an object

\item JavaScript Types are Dynamic:
\begin{verbatim}
var x;           // Now x is undefined
x = 5;           // Now x is a Number
x = "John";      // Now x is a String
\end{verbatim}

\item \textit{avoid} using arrow functions when using \verb|this| in a method, eg.
\begin{verbatim}
let obj = {
  name: 'john',
  sex: 'male',
  speak: () => {
    alert(this.name)
  },
}
obj.speak()
\end{verbatim}
won't alert 'john'

\item Property Value Shorthand:\\
factory function:
\begin{verbatim}
const monsterFactory = (name, age) => {
  return { 
    name: name,
    age: age
  }
};
\end{verbatim}

Imagine if we had to include more properties, that process would quickly become tedious! But we can use a destructuring technique, called \textit{property value shorthand}, to save ourselves some keystrokes
\begin{verbatim}
const monsterFactory = (name, age) => {
  return { 
    name,
    age 
  }
};
\end{verbatim}

\item Destructured Assignment
\begin{verbatim}
const vampire = {
  name: 'Dracula',
  residence: 'Transylvania',
  preferences: {
    day: 'stay inside',
    night: 'satisfy appetite'
  }
};

const { residence } = vampire; 
console.log(residence); // Prints 'Transylvania'
\end{verbatim}
We can even use destructured assignment to grab nested properties of an object:
\begin{verbatim}
const { day } = vampire.preferences; 
console.log(day); // Prints 'stay inside'
\end{verbatim}

\item Built-in Object Methods

object instance methods like: \verb|.hasOwnProperty()|, \verb|.valueOf()|

Object class methods such as \verb|Object.assign()|, \verb|Object.entries()|, and\\ \verb|Object.keys()|

\item \href{https://developer.mozilla.org/en-US/docs/Web/JavaScript/Reference/Global_Objects/Math/round#Description}{Description}: Because \verb|round()| is a static method of \verb|Math|, you always use it as \verb|Math.round()|, rather than as a method of a \verb|Math| object you created (\verb|Math| has no constructor).

\verb|Math| object has no constructor, so there's no instance method (method that starts with \verb|Math.prototype.|). On the other hand, \verb|Number| object has a constructor, it has both instance methods and static methods

\end{enumerate}

\subsubsection{caveats}
\begin{itemize}
\item 取对象值的时候方括号里2和'2'是一样的:
\begin{verbatim}
const obj = {1: "体姿体态", 2: "主要关节活动度", 3: "运动模式", 4: "临床骨科检查", 5: "自定义模式"};
obj[2] === obj['2'] 为 true
\end{verbatim}

\item 

\end{itemize}

\subsubsection{Looping Through Objects}
JavaScript has given us alternative solution for iterating through objects with the \verb|for...in| syntax

If \verb|tickets| is an \textit{empty} object, there will be no console log
\begin{verbatim}
for (let key in tickets) {
  console.log('123');
}
\end{verbatim}

\subsection{Difference between for...of and for...in}
See \url{https://developer.mozilla.org/en-US/docs/Web/JavaScript/Reference/Statements/for...of#Difference_between_for...of_and_for...in}

当单独说对象property的时候应该是指name

\subsection{function}
\begin{itemize}
\item Anonymous function expression (including \verb|var decl., assignmt|)
\begin{verbatim}
var anonymous = function () {
  console.log('anonymous function expression');
};
\end{verbatim}
function expressions that are assigned to variables are not hoisted

\item Named function expression (including \verb|var decl., assignmt|)
\begin{verbatim}
var named = function superPower() {
  console.log('Flying');
};
\end{verbatim}
function expressions that are assigned to variables are not hoisted

\item Function declaration
\begin{verbatim}
function superPower() {
  console.log('Flying');
}
\end{verbatim}

\end{itemize}
Function declaration and function expressions look identical. Anonymous function (declaration or expression) can be passed anywhere that accepts function, but it can't exist on its own


\subsection{class}
\begin{enumerate}

\item Although the subclass automatically inherits the parent methods, you need to use the \verb|super| keyword to set the parent properties

\item static methods: The \verb|.now()| method is static, so you can call it directly from the class, but not from an instance of the class
\begin{verbatim}
static generateName() {
  const names = ['Angel', 'Spike', 'Buffy', 'Willow', 'Tara'];
  const randomNumber = Math.floor(Math.random()*5);
  return names[randomNumber];
}
\end{verbatim}

\item In a \verb|constructor()|, you must always call the \verb|super| method before you can use the \verb|this| keyword — if you do not, JavaScript will throw a reference error, it is best practice to call super on the first line of subclass constructors

\item When multiple classes share properties or methods, they become candidates for \textit{inheritance} — a tool developers use to decrease the amount of code they need to write

\item Notice, we also prepended our property names with underscores (\verb|_name| and \verb|_behavior|), which indicate these properties should not be accessed directly

\end{enumerate}

\subsection{Request}
\begin{enumerate}
\item GET and POST requests can be created a variety of ways
\item Use AJAX to asynchronously request data from APIs. \verb|fetch()| and \verb|async/await| are new functionalities developed in ES6 (promises) and ES8 respectively
\item Promises are a new type of JavaScript object that represent data that will eventually be returned from a request
\item \verb|fetch()| is a web API that can be used to create requests. \verb|fetch()| will return promises
\item We can chain \verb|.then()| methods to handle promises returned by \verb|fetch()|
\item The \verb|.json()| method converts a returned promise to a JSON object
\item \verb|async| is a keyword that is used to create functions that will return promises
\item \verb|await| is a keyword that is used to tell a program to continue moving through the message queue while a promise resolves
\item \verb|await| can only be used within functions declared with \verb|async|
\end{enumerate}

\subsection{Errors}
Errors will prevent a program from executing unless it is handled

\subsection{Promise}
\begin{enumerate}

\item Promises are objects that represent the eventual outcome of an asynchronous operation

\item A Promise object can be in one of three states:
\begin{itemize}
\item Pending
\item Fulfilled
\item Rejected
\end{itemize}

\item We refer to a promise as settled if it is no longer pending -- it is either fulfilled or rejected

\item \begin{verbatim}
const prom = new Promise((resolve, reject) => {
  resolve('Yay!');
});

const handleSuccess = (resolvedValue) => {
  console.log(resolvedValue);
};

prom.then(handleSuccess); // Prints: 'Yay!'
\end{verbatim}

\item Using \verb|catch()| with Promises
\begin{verbatim}
prom
  .then((resolvedValue) => {
    console.log(resolvedValue);
  })
  .then(null, (rejectionReason) => {
    console.log(rejectionReason);
  });
\end{verbatim}
Since JavaScript doesn't mind whitespace, we follow a common convention of putting each part of this chain on a new line to make it easier to read. To create even more readable code, we can use a different promise function: \verb|.catch()|.
\begin{verbatim}
prom
  .then((resolvedValue) => {
    console.log(resolvedValue);
  })
  .catch((rejectionReason) => {
    console.log(rejectionReason);
  });
\end{verbatim}

Correct! \verb|.catch(onReject)| is syntactic sugar for \verb|.then(undefined, onReject)|.

\item Chaining multiple promises see javascript.pdf for a demo

\item Avoiding Common Mistakes
\begin{itemize}
\item Mistake 1: Nesting promises instead of chaining them.
\item Mistake 2: Forgetting to \verb|return| a promise.
\end{itemize}

\item Using Promise.all()

To maximize efficiency we should use concurrency, multiple asynchronous operations happening together. With promises, we can do this with the function \verb|Promise.all()|

\item A Promise's constructor has a single parameter, called the "executor function". The executor function has two parameters -- resolve and reject (array method 'reduce' has a reducer function \url{https://developer.mozilla.org/en-US/docs/Web/JavaScript/Reference/Global_Objects/Array/Reduce}).

\end{enumerate}

\subsection{Async await}
\subsubsection{The async Keyword}
The \verb|async| keyword is used to write functions that handle asynchronous actions. We wrap our asynchronous logic inside a function prepended with the \verb|async| keyword. Then, we invoke that function
\begin{verbatim}
async function myFunc() {
  // Function body here
};

myFunc();
\end{verbatim}
we can also create \verb|async| function expressions
\begin{verbatim}
const myFunc = async () => {
  // Function body here
};

myFunc();
\end{verbatim}
\verb|async| functions always return a promise. This means we can use traditional promise syntax, like \verb|.then()| and \verb|.catch| with our \verb|async| functions. An \verb|async| function will return in one of three ways
\begin{itemize}
\item If there's nothing returned from the function, it will return a promise with a resolved value of \verb|undefined|
\item If there's a non-promise value returned from the function, it will return a promise resolved to that value
\begin{verbatim}
async function fivePromise() { 
  return 5;
}

fivePromise()
.then(resolvedValue => {
    console.log(resolvedValue);
  })  // Prints 5
\end{verbatim}

\item If a promise is returned from the function, it will simply return that promise
\end{itemize}

\subsubsection{The await Operator}
\verb|async| functions are almost always used with the additional keyword \verb|await| inside the function body.

The \verb|await| keyword can only be used inside an \verb|async| function. \verb|await| is an operator: it returns the \textbf{resolved value} of a promise. Since promises resolve in an indeterminate amount of time, \verb|await| halts, or pauses, the execution of our \verb|async| function until a given promise is resolved.
\begin{verbatim}
async function asyncPromAll() {
  const resultArray = await Promise.all([asyncTask1(), asyncTask2(), 
asyncTask3(), asyncTask4()]);
  for (let i = 0; i<resultArray.length; i++){
    console.log(resultArray[i]); 
  }
}
\end{verbatim}

\subsection{axios - npm}
\begin{itemize}
\item Showing a resource\\
\verb|fetch('https://jsonplaceholder.typicode.com/posts/1')|

\item Listing resources\\
\verb|fetch('https://jsonplaceholder.typicode.com/posts')|

\item Creating a resource
\begin{verbatim}
// POST adds a random id to the object sent
fetch('https://jsonplaceholder.typicode.com/posts', {
  method: 'POST',
  body: JSON.stringify({
    title: 'foo',
    body: 'bar',
    userId: 1
  }),
  headers: {
    "Content-type": "application/json; charset=UTF-8"
  }
})
\end{verbatim}

\item Updating a resource
\begin{verbatim}
fetch('https://jsonplaceholder.typicode.com/posts/1', {
  method: 'PUT',
  body: JSON.stringify({
    id: 1,
    title: 'foo',
    body: 'bar',
    userId: 1
  }),
  headers: {
    "Content-type": "application/json; charset=UTF-8"
  }
})
\end{verbatim}
or
\begin{verbatim}
fetch('https://jsonplaceholder.typicode.com/posts/1', {
  method: 'PATCH',
  body: JSON.stringify({
    title: 'foo'
  }),
  headers: {
    "Content-type": "application/json; charset=UTF-8"
  }
})
\end{verbatim}

\item Deleting a resource
\begin{verbatim}
fetch('https://jsonplaceholder.typicode.com/posts/1', {
  method: 'DELETE'
})
\end{verbatim}

\item Filtering resources

Basic filtering is supported through query parameters.
\begin{verbatim}
// Will return all the posts that belong to the first user
fetch('https://jsonplaceholder.typicode.com/posts?userId=1')
\end{verbatim}

\item Nested resources

One level of nested route is available.
\begin{verbatim}
// equivalent to /comments?postId=1
fetch('https://jsonplaceholder.typicode.com/posts/1/comments')
\end{verbatim}


\end{itemize}

\subsection{beforeunload, unload, load}
\begin{table}[h]
\begin{tabular}{|lccc|}
\hline
 & beforeunload & unload & load \\
\hline
刷新 & 1 & 2 & 3 \\
\hline
关闭 & 1 & 2 &  \\
\hline
首次打开 &  &  & 1 \\
\hline
\end{tabular}
\end{table}

\begin{enumerate}
\item 同一浏览器不允许两个及以上账号登录

\item 关闭最后一个标签或者浏览器的时候就默认logout

\item 登录一个账号后,复制某个页面url然后新打开一个标签粘贴进去直接进入到相应的界面

\item 不管开多少个浏览器,只要有一个账号登录则其他浏览器提示不能登录或者前一个登录无效 (同一电脑上面的不同浏览器或者不同电脑上面的浏览器)
\end{enumerate}

\subsection{CORS}
OPTIONS is an HTTP/1.1 method that is used to determine further information from servers, and is a safe method, meaning that it can't be used to change the resource.

Line 7 shows the flag on XMLHttpRequest that has to be set in order to make the invocation with Cookies, namely the withCredentials boolean value.


\section{\href{https://javascript.info/}{javascript.info}}
\subsection{JavaScript Fundamentals}
\subsubsection{Hello, world! (finished including tasks)}
\subsubsection{Code structure (finished)}
\subsubsection{The modern mode, "use strict" (finished)}
When you use a developer console to run code, please note that it doesn't use strict by default.

Modern JavaScript supports “classes” and “modules” \textemdash\ advanced language structures (we'll surely get to them), that enable \verb|use strict| automatically. So we don't need to add the \verb|"use strict"| directive, if we use them.

\subsubsection{Variables (finished including tasks)}

\subsubsection{Data types (finished including tasks)}
\begin{itemize}
\item \textbf{The “null” value}: It's just a special value which represents “nothing”, “empty” or “value unknown”.

\item \textbf{The “undefined” value}: The meaning of undefined is “value is not assigned”.

\end{itemize}

\subsubsection{Interaction: alert, prompt, confirm (finished including tasks)}
\subsubsection{Type Conversions (finished)}
\subsubsection{Basic operators, maths (finished including tasks)}
\verb|null + 1 = 1|, \verb|null| becomes \verb|0| after the numeric conversion.

\subsubsection{Comparisons (finished including task)}
operands of different types are converted to numbers by the equality operator ==.

On the other hand, the equality check \verb|==| for \verb|undefined| and \verb|null| is defined such that, without any conversions, they equal each other and don't equal anything else. That's why (2) \verb|null == 0| is false.

\subsubsection{Conditional operators: if, '?' (finished including tasks)}
\begin{verbatim}
let userInput = prompt(`what's the "official" name of JavaScript?`, ``)

if (userInput && userInput.toLowerCase() === 'ecmascript') {
  alert(`Right`)
} else {
  alert(`You don't know? "ECMAScript"!`)
}
\end{verbatim}
当用户不输入任何值的时候,userInput 为 null,照说 '===' 应该先执行报错(优先级更高),但是没有,为什么?see \href{https://developer.mozilla.org/en-US/docs/Web/JavaScript/Reference/Operators/Operator_Precedence#Note_on_grouping_and_short-circuiting}{Note on Grouping and Short-Circuiting}

\subsubsection{Logical operators (finished including tasks)}
Task: What will the code below output?
\begin{verbatim}
alert( alert(1) || 2 || alert(3) );
\end{verbatim}
The answer: first 1, then 2.

The call to \verb|alert| does not return a value. Or, in other words, it returns \verb|undefined|. see pic null\&undefined.png

\subsubsection{Nullish coalescing operator '??' (finished)}
Summary section is very good.

see pic nullish.png.

\subsubsection{Loops: while and for (finished including tasks)}
A label is the only way for \verb|break/continue| to escape a nested loop to go to an outer one.

\href{https://javascript.info/while-for#output-prime-numbers}{Output prime numbers}

\subsubsection{The "switch" statement (finished including tasks)}

\subsubsection{Functions (finished including tasks)}
If a same-named variable is declared inside the function then it shadows the outer one.

A function with an empty \verb|return| or without it returns \verb|undefined|

Naming a function:
\begin{itemize}
\item "get..." – return a value,
\item "calc..." – calculate something,
\item "create..." – create something,
\item "check..." – check something and return a boolean, etc.
\end{itemize}
\begin{enumerate}
\item showMessage(..)     // shows a message
\item getAge(..)          // returns the age (gets it somehow)
\item calcSum(..)         // calculates a sum and returns the result
\item createForm(..)      // creates a form (and usually returns it)
\item checkPermission(..) // checks a permission, returns true/false
\end{enumerate}

\subsubsection{Function expressions (finished)}
\begin{itemize}
\item Function Declarations are processed before the code block is executed. They are visible everywhere in the block.

\item Function Expressions are created when the execution flow reaches them.
\begin{verbatim}
let age = prompt("What is your age?", 18);

let welcome = (age < 18) ?
  function() { alert("Hello!"); } :
  function() { alert("Greetings!"); };

welcome(); // ok now
\end{verbatim}
that's why the above works as expected

\end{itemize}

\subsubsection{Arrow functions, the basics (finished including tasks)}

\subsubsection{JavaScript specials (finished)}

\subsection{Code quality}
\subsubsection{Debugging in Chrome (finished)}

\subsubsection{Coding Style (finished including task)}
Indents: Spaces are more common nowadays.

In the example below, the initialization of variables, the main loop and returning the result are split vertically:
\begin{verbatim}
function pow(x, n) {
  let result = 1;
  //              <--
  for (let i = 0; i < n; i++) {
    result *= x;
  }
  //              <--
  return result;
}
\end{verbatim}

\href{https://javascript.info/coding-style#automated-linters}{Automated Linters} sheds some info on eslint config

\subsubsection{Comments (finished)}

\subsubsection{Ninja code (finished)}

\subsubsection{Automated testing with Mocha (skipped)}

\subsubsection{Polyfills (finished)}

\subsection{Objects: the basics}
\subsubsection{Objects (finished including tasks)}
\href{https://javascript.info/object#property-names-limitations}{Property names limitations}:
\begin{itemize}
\item In short, there are no limitations on property names. They can be any strings or symbols (a special type for identifiers, to be covered later).

\item Other types are automatically converted to strings.

\end{itemize}

Task: Check for emptiness:
\begin{verbatim}
function isEmpty(obj) {
  for (let key in obj) {
    // if the loop has started, there is a property
    return false;
  }
  return true;
}
\end{verbatim}

\subsubsection{Object copying, references (finished)}
To make a “real copy” (a clone) we can use \verb|Object.assign| for the so-called “shallow copy” (nested objects are copied by reference) or a “deep cloning” function, such as \verb|_.cloneDeep(obj)|.

\subsubsection{Garbage collection (finished)}
Outgoing references do not matter. Only incoming ones can make an object reachable.

Being referenced is not the same as being reachable (from a root): a pack of interlinked objects can become unreachable as a whole.

\subsubsection{Object methods, "this" (finished including tasks)}
\begin{verbatim}
function sayHi() {
  alert(this);
}

sayHi(); // undefined, because it is called as a function, not as a method with “dot” syntax
\end{verbatim}
In this case this is undefined in strict mode. If we try to access this.name, there will be an error.

\noindent In non-strict mode the value of \verb|this| in such case will be the global object (\verb|window| in a browser, we'll get to it later in the chapter Global object). This is a historical behavior that \verb|"use strict"| fixes.

Please note that arrow functions are special: they have no this. When this is accessed inside an arrow function, it is taken from outside.

Task \href{https://javascript.info/object-methods#chaining}{Chaining} contains tests

\subsubsection{Constructor, operator "new" (finished including tasks)}
part \href{https://javascript.info/constructor-new#constructor-mode-test-new-target}{Constructor mode test: new.target} skipped intentionally

\subsubsection{Optional chaining '?.' (finished)}
Further in this article, for brevity, we'll be saying that something “exists” if it's not \verb|null| and not \verb|undefined|.

\subsubsection{Symbol type (finished)}
\begin{itemize}
\item 
A normal symbol
\begin{verbatim}
// id is a new symbol
let id = Symbol();
\end{verbatim}

\item 
Global symbols: \verb|Symbol.for(key)| read (create if absent) a symbol from the registry

\verb|Symbol.keyFor(sym)| returns a name by a global symbol and doesn't work for non-global symbols. If the symbol is not global, it won't be able to find it and returns \verb|undefined|.

\end{itemize}

\subsection{Advanced working with functions}
\subsubsection{Variable scope (finished including tasks)}
Task \href{https://javascript.info/closure#filter-through-function}{Filter through function} may be useful to make note here

\noindent Often used glossaries:
\begin{verbatim}
global scope, block scope, function scope

creation place, invocation place, lexical environment
outer variable, inner variable, enclosing lexical context
\end{verbatim}

\subsection{Modules}
\subsubsection{Modules, introduction (finished)}
Modules work only via HTTP(s), not in local files

\href{https://developer.mozilla.org/en-US/docs/Web/JavaScript/Guide/Modules}{JavaScript modules}: You can only use import and export statements inside modules, not regular scripts.

\subsection{Introduction to Events}
\subsubsection{Introduction to browser events (finished including tasks)}

\href{https://javascript.info/introduction-browser-events#move-the-ball-across-the-field}{Move the ball across the field}

\href{https://javascript.info/introduction-browser-events#carousel}{Carousel}

\subsubsection{Bubbling and capturing (finished)}
The most deeply nested element that caused the event is called a target element, accessible as \verb|event.target|

\subsubsection{Event delegation (finished including tasks)}

\href{https://javascript.info/event-delegation#sortable-table}{Sortable table} (finished with one question)

\verb|let rowsArray = Array.from(tbody.rows)|

will subsequent changes to \verb|rowsArray| object affect \verb|tbody.rows| object? seems like

\href{https://javascript.info/event-delegation#tooltip-behavior}{Tooltip behavior} (finished)

don't create the element and assign it to a variable in global scope, but can declare a variable in the global scope and then create the element in a function scope

\subsubsection{Browser default actions (finished including tasks)}
\textbf{Returning false from a handler is an exception}

The value returned by an event handler is usually ignored.

The only exception is \verb|return false| from a handler assigned using \verb|on<event>|.

In all other cases, \verb|return| value is ignored. In particular, there's no sense in returning \verb|true|.

\subsubsection{Dispatching custom events (finished)}



\section{Web API}
\verb|document.getElementById()| returns an HTMLElement Object which inherits from Element object.


\section{DOM}
首先找到 nodeType,比如a标签的nodeType为1(ELEMENT\_NODE),是 HTMLAnchorElement 元素(怎么直接获取 return HTMLAnchorElement?),继承 HTMLElement \textgreater\ Element \textgreater\ Node \textgreater\ Object (\textgreater\ indicates ``inherited from").\\ see \href{https://developer.mozilla.org/en-US/docs/Web/API/Node/nodeType}{Node.nodeType}

{\color{red}\verb|document.querySelector('a').nodeName| (or \verb|tagName|)} will return 'A', which means HTMLAnchorElement. see \href{https://developer.mozilla.org/en-US/docs/Web/API/Node/nodeName}{Node.nodeName}

\verb|document.querySelector('a').nodeValue| will return \verb|null|, because \verb|nodeValue| of \verb|Element| is \verb|null|. see \href{https://developer.mozilla.org/en-US/docs/Web/API/Node/nodeValue}{Node.nodeValue}

Answer following questions (what will be the result?)
\begin{verbatim}
<a href="#">Hi</a>

<script>
  console.log(document.querySelector('a').firstElementChild.nodeValue);
</script>
\end{verbatim}

it will throw an exception, because there's no \verb|ELEMENT_NODE| nodeType child for the anchor element (obtained through \verb|document.querySelector('a')|), it only has one \verb|TEXT_NODE| nodeType child. So, \verb|document.querySelector('a').firstElementChild| will return \verb|null|. \verb|document.querySelector('a').firstChild| will return a \verb|TEXT_NODE| nodeType child, because the text 'Hi' is enclosed in a \verb|TEXT_NODE| which is then enclosed in \verb|HTMLAnchorElement|. see \href{https://developer.mozilla.org/en-US/docs/Web/API/Node/firstChild}{Node.firstChild}

\begin{enumerate}
\item \verb|document.body| returns body element (a nodeType of \verb|ELEMENT_NODE|)

\item 在查阅MDN上的文档时,比如 \verb|HTMLTextAreaElement|,左侧显示 Properties 只有一个,是显示不完整,正文当中是显示完整的

\end{enumerate}


\section{CSS}

\subsection{Questions}
当用浏览器检查页面元素时,可以看到 background 图片的路径与代码里面写的不一样,多了图片名和图片格式之间的字母,为什么?(可以参考 CSSbackgroundImgPathInBrowsers.png)

\subsection{Caveats}
\begin{enumerate}

\item {\color{red}\verb|box-sizing: border-box;|} 则width=border width + padding width + content (area) width. chrome inspect的width、height总是这些值之和

\item The unit px, if the value is 0 then you can omit the px unit, if it's not 0 then you have to add px

\end{enumerate}


\subsection{CSS Typography}
\begin{enumerate}

\item Serif fonts are widely used for body text (eg. articles) because they are considered easier to read than sans-serif fonts in print.

\item You can look up the font you are using to see which \verb|font-weight| values are available

\item The technical term for adjusting the spacing between letters is called ``kerning''. Kerning can be adjusted with the \verb|letter-spacing| property in CSS

\item \verb|line-height|: Generally, the unitless ratio value is the preferred method, since it is responsive and based exclusively on the current font size. In other words, if we change the font size, a unitless \verb|line-height| would automatically readjust, whereas the pixel value would remain static

\end{enumerate}

\subsubsection{Fallback Fonts}
To use fallback fonts, the following syntax is required:
\begin{verbatim}
h1 {
  font-family: "Garamond", "Times", sans-serif;
}
\end{verbatim}
The CSS rule above says:
\begin{enumerate}
\item Use the Garamond font for all \verb|<h1>| elements on the web page

\item If Garamond is not available, use the Times font

\item If Garamond and Times are not available, use any serif font pre-installed on the user's computer

\end{enumerate}

\subsubsection{Linking fonts or font-face}
\begin{itemize}
\item Linking fonts: inside head tag in html file
\begin{verbatim}
<head>
  <link href="https://fonts.googleapis.com/css?family=Droid+Serif|Playfair+Display" type="text/css" rel="stylesheet">
</head>
\end{verbatim}

\item Above link fonts can be imported using the \verb|@font-face| property, just copy the url into the browser and then copy the rules that are directly labeled as \verb|/* latin */|

\item It is important to stress the need to copy the \verb|@font-face| rules to the top of the stylesheet for the font to load correctly in the project

\item Using \verb|@font-face| rule to use local font files. See pic \verb|@font-face| for examples of both local files and directly putting linked fonts into css files, see \href{https://developer.mozilla.org/en-US/docs/Web/CSS/@font-face}{@font-face} for format types inside the \verb|format()| function (\textbf{Note:} you can omit the \verb|format()| part, but if you provide one it must be the correct corresponding type)
\end{itemize}



\subsection{Media Queries}
\begin{itemize}
\item 

\item Rather than set breakpoints based on specific devices (too many different device screen sizes), the best practice is to resize your browser to view where the website naturally breaks based on its content. The dimensions at which the layout breaks or looks odd become your media query breakpoints. Within those breakpoints, we can adjust the CSS to make the page resize and reorganize

\item The points at which media queries are set are called breakpoints. For example, if we want to target tablets that are in landscape orientation, we can create the following breakpoint:
\begin{verbatim}
@media only screen and (min-width: 768px) and (max-width: 1024px) and 
(orientation: landscape) {
    /* CSS ruleset */
}
\end{verbatim}
\verb|and (min-width: 768px)| \textemdash This part of the rule is called a \textit{media feature}, and instructs the CSS compiler to apply the CSS styles to devices with a width of 768px or larger

\end{itemize}

\subsection{CSS Display and Positioning}
\begin{itemize}
\item 即使position设置为absolute,其宽度用百分比表示时是相对于父节点的宽度而言,见\ref{position}

\item Floated elements must have a width specified

\item \verb|Float| works for static and relative positioned elements

\item  If you're simply interested in moving an element as far left or as far right as possible on the page, you can use the \verb|float| property

\end{itemize}

Difference in text wraping between \verb|display: inline-block;| and \verb|display: inline;| see pics inline-block.png and inline.png

\subsection{The box model}
\begin{itemize}

\item All major web browsers have a default stylesheet they use in the absence of an external stylesheet. These default stylesheets are known as \textit{user agent stylesheets}. In this case, the term "user agent" is a technical term for the browser

\item The overflow property is set on a parent element to instruct a web browser how to render child elements. For example, if a div's overflow property is set to \verb|scroll|, all children of this div will display overflowing content with a scroll bar.

\item Margin collapse: Unlike horizontal margins, vertical margins do not add. Instead, the larger of the two vertical margins sets the distance between adjacent elements (see CSS Mastery, 3rd Edition, Chapter 3, margin collapsing subsection)

\item The \verb|padding| property is often used to expand the background color and make content look less cramped

\item By default, the dimensions of an HTML box are set to hold the raw contents of the box

\end{itemize}

\subsection{containing blocks}
The process for identifying the containing block depends entirely on the value of the element's \verb|position| property, see \href{https://developer.mozilla.org/en-US/docs/Web/CSS/Containing_block#Identifying_the_containing_block}{Identifying the containing block}, also see \href{https://developer.mozilla.org/en-US/docs/Web/HTML/Block-level_elements}{Block-level elements}

A block container could be block, inline-block or list-item element, the containing block could be a block container or which establishes a formatting context

block container, block-level elements, inline elements, containing block, \href{https://developer.mozilla.org/en-US/docs/Web/Guide/CSS/Block_formatting_context}{(block) formatting context}, viewport

\subsection{Non-Inherited properties}
To name but a few, non-inherited properties are:\\ \verb|width, height, padding, border, margin, position, background|, etc.

\subsection{Learn Responsive Design}
\begin{enumerate}
\item \textbf{Historically}, the em represented the width of a capital letter M in the typeface and size being used. That is no longer the case. Today, the em represents the size of the base font being used. For example, if the base font of a browser is 16 pixels (which is normally the default size of text in a browser), then 1 em is equal to 16 pixels. 2 ems would equal 32 pixels, and so on

\item A high resolution display may have a \verb|min-resolution| of \verb|150dpi|

\item The height property is set to auto, meaning an image's height will automatically scale proportionally with the width

the last line will display images as block level elements (rather than inline-block, their default state)
\begin{verbatim}
.container {
  width: 50%;
  height: 200px;
  overflow: hidden;
}

.container img {
  max-width: 100%;
  height: auto;
  display: block;
}
\end{verbatim}
Images or videos  will shrink to the full width of their container, scale proportionally, and display partially if the image dimensions exceed container dimensions

It's worth memorizing the entire example above. It represents a very common design pattern used to scale images and videos proportionally

\item Percentages are often used to size box-model values, like width and height, padding, border, and margins. They can also be used to set positioning properties (top, bottom, left, right). When percentages are used, elements are sized relative to the dimensions of their parent element (also known as a container) \label{position}\\
\textbf{Note:} Because the box model includes padding, borders, and margins, setting an element's width to 100\% may cause content to overflow its parent container. While tempting, 100\% should only be used when content will not have padding, border, or margin

\item Rem stands for root em. It acts similar to em, but instead of checking parent elements to size font, it checks the root element. The root element is the \verb|<html>| tag

\item \begin{verbatim}
.splash-section {
  font-size: 18px;
}

.splash-section h1 {
  font-size: 1.5em;
}
\end{verbatim}
Instead, a base font size (\verb|18px|) is defined for all text within the \verb|splash-section| element. The second CSS rule will set the font size of all \verb|h1| elements inside of \verb|splash-section| relative to the base font of \verb|splash-section| (18 pixels). The resulting font size of \verb|h1| elements will be 27 pixels.
\end{enumerate}

\subsubsection{Percentages: Padding \& Margin}
\begin{enumerate}
\item When percentages are used to set padding and margin, however, they are calculated based only on the \textit{width} of the parent element.

For example, when a property like \verb|margin-left| is set using a percentage (say \verb|50%|), the element will be moved halfway to the right in the parent container (as opposed to the child element receiving a margin half of its parent's margin)

Vertical padding and margin are also calculated based on the width of the parent

\end{enumerate}

\subsection{display property}
\textbf{Note:} Setting the display property of an element only changes \textbf{how the element is displayed}, NOT what kind of element it is. So, an inline element with \verb|display: block;| is not allowed to have other block elements inside it.

\subsection{Grid layout}
\begin{enumerate}
\item flexbox is really meant for only specific use cases, like navigation bars (navbars)

\item \begin{verbatim}
.grid {
  display: grid;
  width: 1000px;
  height: 500px;
  grid-template-columns: 100px 200px;
  grid-template-rows: 10% 20% 600px;
}
\end{verbatim}
This grid has two columns and three rows. \verb|grid-template-rows| defines the number of rows and sets each row's height. In this example, the first row is 50 pixels tall (10\% of 500), the second row is 100 pixels tall (20\% of 500), and the third row is 600 pixels tall

When using percentages in these two properties, remember that rows are defined as a percentage of the grid's height, and columns are defined as a percentage of its width.

\item The property \verb|grid-template| can replace the previous two CSS properties\\
\verb|grid-template: 200px 300px / 20% 10% 70%;|\\
When using \verb|grid-template|, the values before the slash will determine the size of each row. The values after the slash determine the size of each column

\end{enumerate}

\subsubsection{Fraction}
You may already be familiar with several types of responsive units such as percentages (\verb|%|), \verb|em|s and \verb|rem|s. CSS Grid introduced a new relative sizing unit -- \verb|fr|, like fraction.\\
\verb|grid-template-columns: 1fr 60px 1fr;|\\
In this example, 60 pixels are taken up by the second column. Therefore the first and third columns have 40 available to split between them. Since each gets one fraction of the total, they both end up being 20 pixels wide

\subsubsection{Repeat}
The properties that define the number of rows and columns in a grid can take a function as a value. \verb|repeat()| is one of these functions. The \verb|repeat()| function was created specifically for CSS Grid\\
\verb|grid-template-columns: repeat(3, 1fr);| is the same as writing\\ \verb|grid-template-columns: 1fr 1fr 1fr;|

Finally, the second parameter of repeat() can have multiple values.\\
\verb|grid-template-columns: repeat(2, 20px 50px)|\\
This code will create four columns where the first and third columns will be 20 pixels wide and the second and fourth will be 50 pixels wide

\subsubsection{minmax}
\begin{verbatim}
.grid {
  display: grid;
  grid-template-columns: 100px minmax(100px, 500px) 100px;
}
\end{verbatim}

In this example, the first and third columns will always be 100 pixels wide, no matter the size of the grid. The second column, however, will vary in size as the overall grid resizes. The second column will always be between 100 and 500 pixels wide

\subsubsection{Grid gap}
The CSS properties \verb|grid-row-gap| and \verb|grid-column-gap| will put blank space between every row and column in the grid
\begin{verbatim}
.grid { 
  display: grid; 
  width: 320px; 
  grid-template-columns: repeat(3, 1fr); 
  grid-column-gap: 10px; 
}
\end{verbatim}

It is important to note that \verb|grid-gap| does not add space at the beginning or end of the grid. In the example code, our grid will have three columns with two ten-pixel gaps between them

Let's quickly calculate how wide these columns are. Remember that using \verb|fr| considers all of the available space. The grid is 320 pixels wide and 20 of those pixels are taken up by the two grid gaps. Therefore each column takes a piece of the 300 available pixels. Each column gets 1\verb|fr|, so the columns are evenly divided into thirds (or 100 pixels each)

Finally, there is a CSS property \verb|grid-gap| that can set the row and column gap at the same time. \verb|grid-gap: 20px 10px;| will set the distance between rows to 20 pixels and the distance between columns to 10 pixels. Unlike other CSS grid properties, this shorthand does not take a \verb|/| between values! If only one value is given, it will set the column gap and the row gap to that value

But what if we don't want there to be a gap between all of our rows? We can avoid the \verb|grid-row-gap| rule by using negative margins. In \verb|header|, \verb|.banner|, and \verb|.about|, set \verb|margin-bottom| to \verb|-20px|. For the footer, set \verb|margin-top| to \verb|-20px|.

\subsubsection{Multiple Row and Column Items}
We are no longer applying CSS to the outer grid container; we're adding CSS to the elements sitting inside the grid!
\begin{verbatim}
.item {
  grid-row-start: 1;
  grid-row-end: 3;
}
\end{verbatim}
In this example, the HTML element of class item will take up two rows in the grid, rows 1 and 2. The values that \verb|grid-row-start| and \verb|grid-row-end| accept are \textit{grid lines}.

Row grid lines and column grid lines start at 1 and end at a value that is 1 greater than the number of rows or columns the grid has. For example, if a grid has 5 rows, the grid row lines range from 1 to 6. If a grid has 8 columns, the grid row lines range from 1 to 9.

The value for \verb|grid-row-start| should be the row at which you want the grid item to begin. The value for \verb|grid-row-end| should be one greater than the row at which you want the grid item to end. An element that covers rows 2, 3, and 4 should have these declarations: \verb|grid-row-start: 2| and \verb|grid-row-end: 5|.

We can use the property \verb|grid-row| as shorthand for \verb|grid-row-start| and \verb|grid-row-end|. The following two code blocks will produce the same output:
\begin{verbatim}
.item {
  grid-row-start: 4;
  grid-row-end: 6;
}
\end{verbatim}

\begin{verbatim}
.item {
  grid-row: 4 / 6;
}
\end{verbatim}
This code should look similar to the way \verb|grid-template| is shorthand for \verb|grid-template-rows| and \verb|grid-template-columns|. In this case, the starting row goes before the ``\verb|/|'' and the ending row goes after it. Again, the ending row is exclusive; this grid item will occupy rows four and five.

When an item spans multiple rows or columns using these properties, it will also include the \verb|grid-gap| if any exists. For example, if an item spans two rows of height 100 pixels and there is a ten-pixel grid-gap, then the item will have a total height of 210 pixels.

The previous three properties also exist for columns.
\begin{verbatim}
.item {
  grid-column: 4 / span 2;
}
\end{verbatim}
produces the same result as the following code blocks:
\begin{verbatim}
.item {
  grid-column: 4 / 6;
}
\end{verbatim}

\begin{verbatim}
.item {
  grid-column-start: 4;
  grid-column-end: span 2;
}
\end{verbatim}

\begin{verbatim}
.item {
  grid-column-start: span 2;
  grid-column-end: 6;
}
\end{verbatim}

\subsubsection{Grid Area}
We've already been able to use grid-row and grid-column as shorthand for properties like grid-row-start and grid-row-end. We can refactor even more using the property grid-area. This property will set the starting and ending positions for both the rows and columns of an item
\begin{verbatim}
.item {
  grid-area: 2 / 3 / 4 / span 5;
}
\end{verbatim}
grid-area takes four values separated by slashes. The order is important! This is how grid-area will interpret those values
\begin{enumerate}
\item \verb|grid-row-start|
\item \verb|grid-column-start|
\item \verb|grid-row-end|
\item \verb|grid-column-end|
\end{enumerate}

\subsection{Advanced CSS Grid}
\subsubsection{Grid Template Areas}
The \verb|grid-template-areas| property allows you to name sections of your web page to use as values in the \verb|grid-row-start|, \verb|grid-row-end|, \verb|grid-col-start|, \verb|grid-col-end|, and \verb|grid-area| properties. See javascript.pdf for an example

\subsubsection{Overlapping Elements}

\verb|align-items|为什么不在item里面声明,而要在grid containers里面声明。\verb|align-content| 在grid containers里面声明还好理解。\verb|align-self|是在grid items里面声明。(justify-...同此)

\verb|align-items|是对所有grid items。而\verb|align-self|是对单个grid item,优先级大于\verb|align-items|

\subsection{CSS animation}
\url{https://www.youtube.com/watch?v=t2zoXRXpCWs&list=PL4cUxeGkcC9iGYgmEd2dm3zAKzyCGDtM5&index=8}


\subsection{p tag}
\begin{enumerate}
\item \verb|<figcaption><p>The <em>Cosmos</em> is all there is!</p></figcaption>| if you don't add \verb|p| tag then there will be no space before or after \verb|Cosmos|
\end{enumerate}

\subsection{w3schools CSS Advanced}
\noindent finished:
\begin{itemize}
\item CSS Rounded Corners

\item CSS Border Images

\item CSS Backgrounds. 

Full Size Background Image:
\begin{verbatim}
html {
  background: url(img_man.jpg) no-repeat center fixed;
  background-size: cover;
}
\end{verbatim}

\item CSS Colors

\item CSS Gradients

\item CSS Shadows

\item CSS Text Effects

\item CSS Web Fonts

\item CSS 2D Transforms

\item 

\item CSS Transitions. CSS transitions allows you to change property values smoothly, over a given duration.

To create a transition effect, you must specify two things:
\begin{enumerate}
\item the CSS property you want to add an effect to

\item the duration of the effect

\end{enumerate}


\end{itemize}


\section{HTML}
\subsection{table}
Note that the table's head still requires a row in order to contain the table headings.
\begin{verbatim}
<table>
  <thead>
    <tr>
      <th></th>
      <th scope="col">Saturday</th>
      <th scope="col">Sunday</th>
    </tr>
  </thead>
  <tbody>
    <tr>
      <th scope="row">Morning</th>
      <td rowspan="2">Work</td>
      <td rowspan="3">Relax</td>
    </tr>
    <tr>
     <th scope="row">Afternoon</th>
    </tr>
    <tr>
      <th scope="row">Evening</th>
      <td>Dinner</td>
    </tr>
  </tbody>
</table>
\end{verbatim}

\subsection{form}
without the \verb|name| attribute, information in the \verb|<input>| won't be sent when the \verb|<form>| is submitted.


\section{jQuery}
jQuery can equally interpret the CSS and DOM formatting of multiple-word properties. For example, jQuery understands and returns the correct value for both \verb|.css( "background-color" )| and \verb|.css( "backgroundColor" )|. This means mixed case has a special meaning, \verb|.css( "WiDtH" )| won't do the same as \verb|.css( "width" )|, for example.


\section{D3}
\subsection{Introduction to D3 (data-driven styling)}
\begin{enumerate}
\item 
\begin{verbatim}
let dataset = [55,34,23,22,59];
d3.selectAll("p")
   .data(dataset);
\end{verbatim}
Remember that the steps that follow \verb|.data()| method are executed the total number of elements in the dataset.

\item 
After making a selection, creating a new visualization, whose elements are not already in the DOM, usually follows the .data() .enter() and .append() pattern:
\begin{verbatim}
let dataset = [55,34,23,22,59];
d3.select("body")
   .selectAll("div")
   .data(dataset)
   .enter()
   .append("div");
\end{verbatim}

\item You'll see that the d3 community will indent some parts of the chains with four spaces, and other parts of the chain with two. The lines with two spaces indicate that the selection has changed:
\begin{verbatim}
var h1 = d3.selectAll("section")
    .style("background", "steelblue")
  .append("h1")
    .text("Hello!");
\end{verbatim}

\item 

\end{enumerate}


\section{高德地图}
\href{https://lbs.amap.com/api/jsapi-v2/guide/abc/load}{JSAPI 的加载} 按 NPM 方式使用 Loader 时{\color{red}Uncaught SyntaxError: Cannot use import statement outside a module}

\subsection{\href{https://lbs.amap.com/api/jsapi-v2/guide/overlays/marker}{点标记}}
灵活点标记 ElasticMarker 需要学习

documentation看不懂的可以看guide及结合example

\subsection{事件系统}
\subsubsection{地图和覆盖物事件 (finished)}


\section{to learn list}
JSDoc

UML

functional programming language: Scala


\section{Java}
Java + input tag (type hidden) 试着写一个demo练手, 可参考 \url{https://developer.mozilla.org/en-US/docs/Web/HTML/Element/input/hidden#Examples}


\section{设计图}
xiong设计的图里面 opacity 值应由 \verb|rgba()| 的a来代替,可用于 \verb|background-image| 的 \verb|linear-gradient| 方法中,也可用于 \verb|background-color| 中


\section{wechat mini program}

\subsection{\href{https://developers.weixin.qq.com/miniprogram/dev/framework/quickstart/code.html}{小程序代码构成}}
\begin{itemize}
\item \textbf{JSON 语法} 忘了给 Key 值加双引号或者是把双引号写成单引号是常见错误。

还需要注意的是 JSON 文件中无法使用注释,试图添加注释将会引发报错

\item 

\end{itemize}

\subsection{框架}
\begin{enumerate}

\item 开发者需要做的只是将页面的数据、方法、生命周期函数注册到 框架 中

\item 框架的核心是一个响应的数据绑定系统
\item 整个小程序框架系统分为两部分:视图层(View)和逻辑层(App Service)
\end{enumerate}

\vspace{2in}
小程序~服务号~订阅号~企业号


\section{React Native}
\begin{itemize}
\item RN API 的 props 没有 boxSizing 这一属性。用模拟器 Inspect 的时候,padding 和 content合并在一起了,没有合并在一起,可以查看Pic RNSimulatorInspect.png

\item react-devtools 用npm和yarn都安装失败了

\item View标签 (container),不能直接容纳文本,以下会app crash

\verb|<View>票速选择</View>|

React Native styles all views as if box-sizing: \verb|border-box| is set

\item View默认\verb|display: flex;|,Text: everything inside is no longer using the flexbox layout but using text layout.

\item 查找文档的时候,注意版本号是否对应

\item RN的TouchableOpacity没有列出style属性但是例子里面添加了
\end{itemize}

\subsection{Style}
The \verb|style| prop can be a plain old JavaScript object. That's the simplest and what we usually use for example code. You can also pass an array of styles - the last style in the array has precedence, so you can use this to inherit styles

\subsection{Height and Width}
All dimensions in React Native are unitless, and represent density-independent pixels


\section{Node}
\begin{enumerate}
\item Enter repl (read evaluate print loop): type node in command line, exit command \verb|process.exit()|
\item \textbf{How to set Environment variables from within package.json [Node.js]}\\
\verb|"start-production": "SET NODE_ENV=production&&webpack-dev-server --mode production --open"|
\end{enumerate}



\section{npm (or yarn add)}
\begin{enumerate}

\item 

\item If you're using npm 5, you'll probably also see a package-lock.json file in your directory

\item Useful command \verb|npm show create-react-app version|\\
Get list of locally installed packages: \verb|npm list|, you can find the version of a specific package by passing its name as an argument. For example, \verb|npm list grunt|\\
Get list of globally installed packages: \verb|npm list -g --depth 0|

\item nvm: node version manager, to manage multiple node versions on one operating system

\item To update Node, the most reliable way is to download and install an updated installer package from their website (see link above). To update npm, use the following command in your terminal:
\begin{verbatim}
npm install npm@latest -g
\end{verbatim}
\end{enumerate}

\subsection{browser-sync}
\verb|browser-sync start --server -w --startPath="/javascriptInfo.html"|

\section{DOM}
\begin{enumerate}

\item \verb|document.createElement()|
\begin{verbatim}
<!DOCTYPE html>
<html>
<head>
  <title>||Working with elements||</title>
</head>
<body>
  <div id="div1">The text above has been created dynamically.</div>
</body>
</html>

document.body.onload = addElement;

function addElement () { 
  // create a new div element 
  var newDiv = document.createElement("div"); 
  // and give it some content 
  var newContent = document.createTextNode("Hi there and greetings!"); 
  // newDiv.textContent = "Hi there and greetings!".
  
  // add the text node to the newly created div
  newDiv.appendChild(newContent);  

  // add the newly created element and its content into the DOM 
  var currentDiv = document.getElementById("div1"); 
  document.body.insertBefore(newDiv, currentDiv); 
}
\end{verbatim}

\item \verb|Document| and \verb|window| objects are the objects whose interfaces you generally use most often in DOM programming. In simple terms, the \verb|window| object represents something like the browser, and the \verb|document| object is the root of the document itself. \verb|Element| inherits from the generic \verb|Node| interface, and together these two interfaces provide many of the methods and properties you use on individual elements

\item That is to say, it's \textit{written} in JavaScript, but it uses the DOM to access the document and its elements
\end{enumerate}


\section{cmd}
\begin{enumerate}

\item 

\item \href{https://stackoverflow.com/questions/41139067/git-log-output-encoding-issues-on-windows-10-command-prompt}{git log output encoding issues on Windows 10 command prompt}

\item Windows clear screen commands:

cls (clear screen) \quad Press Esc key \quad Ctrl + c

\end{enumerate}


\section{中文笔记}
\begin{itemize}

\item 


\item 变量提升:JS会在词法分析阶段,把变量的声明语句提升到作用域的顶部

\end{itemize}


\section{命名规范 Name convention}
\begin{itemize}
\item 线路首页: Route or RouteHome

\item 

\item 

\end{itemize}



\section{Abbr}
\begin{tabular}{ll@{\hspace{3em}}ll@{\hspace{3em}}ll}
% &  &  &  &  &  \\
block & 块级 & inline & 内联 & lexical & 词法作用域 \\
closure & 闭包 & render & 渲染 & container & 容器 \\
content & 内容 & form & 表单 & canvas & 画布 \\
component & 组件 & margin & 外边距 & style & 样式 \\
callback & 回调 & revert & 回滚 & collapse & 折叠 \\
\multicolumn{3}{l}{template string 模板字符串} & \multicolumn{3}{l}{logical operators 逻辑运算符}\\
\multicolumn{3}{l}{object destructure 对象解构}  & \multicolumn{3}{l}{array destructure 数组解构} \\ 
\multicolumn{3}{l}{logical operands: true false}  & \multicolumn{3}{l}{SoC: Separation of Concerns} \\\multicolumn{6}{l}{POSIX: portable operating system interface for unix } \\
\multicolumn{4}{l}{umd: universal module definition} \\
\multicolumn{4}{l}{XSS attack: cross-site scripting attack} & literal & 字面量 \\
 &  &  &  &  & \\
\end{tabular}



\end{document}
